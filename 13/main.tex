
\title{13 вопрос}
\author{Ансар Каржаспаев @ansark4}

\documentclass[a4paper, 10pt]{article}

\usepackage[T2A]{fontenc}
\usepackage[utf8]{inputenc} 
\usepackage[russian]{babel}
\usepackage{indentfirst}
\usepackage{amssymb}
\usepackage{enumitem}
\usepackage{hyperref}
\usepackage{geometry}
\usepackage{mathtools}
\usepackage{setspace}
\usepackage{tikz}
\geometry{a4paper,top=2cm,bottom=2cm,left=2cm,right=2cm}
\usepackage{HWspecial}

\usepackage{fancyhdr}
\pagestyle{fancy}

\makeatletter
\def\thickhrulefill#1{\leavevmode\leaders\hrule height#1\hfill\kern\z@}
\makeatother

\renewcommand{\headrule}{
    \color{black}\vspace{-8pt}
    \thickhrulefill{.6pt}

    \vspace{-9pt}\thickhrulefill{2.4pt}
}



\begin{document}

\maketitle
\section{Критерий Коши существования предела функции. Односторонние пределы и теорема Вейерштрасса о существовани односторонних пределов монотонной ограниченной функции.
}

\smallskip
\subsection{Критерий Коши существования предела функции.}
Теорема 65 (Критерий Коши). Пусть f: D$\to$ R и a предельная точка D. Предел $\lim_{x \to a} f(x)$ существует тогда и только тогда, когда для каждого $\epsilon$ > 0 найдется такое $\delta$ > 0, что для каждых $x, y \in B^{'}_{\delta} (a) \cap D$  выполнено $|f(x) - f(y)| < \epsilon$. \\
Доказательство. Если $\lim_{x \to a} f(x) = A$, то для каждого $\epsilon > 0$ найдется такое $\delta$ > 0, что для произвольной точки $x \in B^{'}_{\delta} (a) \cap D$  выполнено $|f(x) - A| < \epsilon / 2$. Тогда для произвольных точек $x, y \in B^{'}_{\delta} (a) \cap D$ выполнено $|f(x) - f(y)| \leq |f(x) - A| + |A - f(y)| < \epsilon$. \\

Предположим, что выполнено условие Коши. Тогда для произвольной последовательности точек $x_n \in D \backslash$ \{a\}, $x_n \to a$, последовательность $\{f(x_n)\}$ является фундаментальной, а значит сходится. Пусть $\lim_{x \to \infty} f(x_n) = A$. Если есть другая последовательность точек $y_n \in D \backslash \{a\}, y_n \to a$, то рассмотрим новую последовательность $z_{2k1} = x_k, z_{2k} = y_k$, т.е. эта последовательность вида $x_1, y_1, x_2, y_2, \dots \subset D \backslash \{a\}$. Эта последовательность также сходится к a, поэтому последовательность образов $f(x_1), f(y_1), f(x_2), f(y_2), \dots$ снова оказывается фундаментальной, а потому сходится. В силу того, что предел подпоследовательности сходящейся последовательности совпадает с пределом всей последовательности, получаем, что $\lim_{x \to \infty} f(y_n) = A$. Таким образом, доказано существование предела по Гейне.
\smallskip 
\subsection{Односторонние пределы и теорема Вейерштрасса о существовани односторонних пределов монотонной ограниченной функции.} 
Пусть $D^{+}_{a} := D \cap (a, +\infty)$ и $D^{-}_{a} := D \cap (-\infty, a)$\\
Определение 66. Пусть точка a предельная для множества $D^{+}_{a}$ и существует предел функции f по множеству $D^{+}_{a}$ в точке a. Этот предел называют пределом справа функции f в точке a и обозначают $\lim_{x \to a+0} f(x)$. Аналогично определяется предел слева, который обозначают $\lim_{x \to a-0} f(x)$. \\

Теорема 67 (Вейерштрасс). Пусть f не убывает и ограничена на множестве D, a - предельная точка множества $D^{-}_{a}$. Тогда существует предел слева
$$\lim_{x \to a-0} f(x) = sup\{f(x):x \in D^{-}_{a}\}$$
Пусть f не убывает и ограничена на множестве D, a предельная точка множества $D^{+}_{a}$. $$\lim_{x \to a+0} f(x) = inf\{f(x):x \in D^{+}_{a}\}$$
Аналогичные утверждения с заменой inf на sup справедливы и для невозрастающей
функции.\\
Доказательство. Пусть M = $sup\{f(x):x \in D^{-}_{a}$. Тогда для каждого $\epsilon$ > 0 найдется такая точка $x_0 \in D^{-}_{a}$, что $M-\epsilon < f(x_0)$. Т.к. f не убывает на $D^{-}_{a}$, то для каждого $x \in (x_0, a) \cap D^{'}_{a} $ выполнено $M - \epsilon < f(x_0) \leq f(x) \leq M < M + \epsilon$. Тогда, взяв $\delta := a - x_0$ получаем, что для каждого $x \in B^{'}_{\delta}(a) \cap D^{-}_{a}$ выполнено $|f(x) - M| < \epsilon$.

\end{document}
