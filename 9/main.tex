\documentclass[12pt]{article}
\usepackage[utf8]{inputenc}
\usepackage[russian]{babel}
\usepackage[left=1.5cm,right=1.5cm,top=1.5cm,bottom=2cm]{geometry}
\usepackage{graphicx}
\usepackage{color}
\usepackage{titlesec}
\usepackage{amssymb}
\usepackage{mathtools}
\usepackage{minted}
\usepackage{hyperref}
\usepackage{amsmath}
\usepackage{amsfonts}
\usepackage{fancyhdr}
\usepackage{enumitem}
\usepackage{mathbbol}
\usepackage{titlesec}
\usepackage{changepage}
\usepackage{listings}
\usepackage[titles]{tocloft}
\usepackage{tcolorbox}
\usepackage{tikz}
\usepackage{tikz-cd}
\usepackage{ulem}
\usepackage{nicefrac}
\usepackage{indentfirst}
\usepackage{dsfont}
\usepackage{amsthm}
\usepackage{hyperref}
\hypersetup{
    colorlinks=true,
    linkcolor=cyan,
    filecolor=magenta,      
    urlcolor=blue,
    pdftitle={Overleaf Example},
    pdfpagemode=FullScreen,
    }

\begin{document}

\section{Числовые ряды}

\subsection{Числовой ряд}
Пусть дана последовательность $\textstyle\{a_n\}_{n=1}^{\infty}$, тогда числовым рядом с членами $a_n$ называется выражение:
$$a_1 + a_2 + a_3 + ... = \displaystyle\sum_{k = 1}^{\infty}a_k$$

\subsection{Переформулировка критерия Коши для числовых рядов}
Ряд сходится тогда и только тогда, когда:
$$\forall \varepsilon > 0 \ \exists N \in \mathbb{N} : \forall n > m > N \to \left|\displaystyle\sum_{k = m + 1}^n a_k\right| = |S_n - S_m| < \varepsilon$$

\subsection{Необходимое условие сходимости числового ряда}
Если числовой ряд сходится, то $a_k \to 0$ при $k \to \infty $
\begin{proof}
Из критерия коши следует, что: $$\forall \varepsilon > 0 \ \exists N \in \mathbb{N}: \forall n >  N + 1 \to |a_n| = |S_n - S_{n-1}| < \varepsilon$$
\end{proof}

\subsection{Абсолютная и условная сходимость рядов}
\begin{enumerate}
Говорят, что ряд $\displaystyle\sum_{k = 1}^{\infty}a_k$ сходится абсолютно, если сходится ряд $\displaystyle\sum_{k = 1}^{\infty}|a_k|$ \\
Говорят, что ряд $\displaystyle\sum_{k = 1}^{\infty}a_k$ сходится условно, если он сходится, а ряд $\displaystyle\sum_{k = 1}^{\infty}|a_k|$ расходится 
\end{enumerate}

\subsection{Cходимость абсолютно сходящегося ряда}
Если ряд $\displaystyle\sum_{k = 1}^{\infty}|a_k|$ сходится, то и $\displaystyle\sum_{k = 1}^{\infty}a_k$ тоже сходится
\begin{proof}
Из сходимости ряда $\displaystyle\sum_{k = 1}^{\infty}|a_k|$ следует выполнение критерия Коши для этого ряда, то есть что:
$$\forall \varepsilon > 0 \ \exists N \in \mathbb{N}: \forall n > m > N \to \displaystyle\sum_{k = m + 1}^{n}|a_k| < \varepsilon$$
но так как $\displaystyle\sum_{k = m + 1}^{n}|a_k| \geq \left|\displaystyle\sum_{k = m + 1}^{n}a_k\right|$, то критерий Коши выполнен и для ряда без модулей.
\end{proof}

\subsection{Признак сравнения}
\begin{enumerate}
Пусть $0 \leq a_n \leq b_n$ тогда если ряд $\displaystyle\sum_{k = 1}^{\infty}b_k$ сходится, то и $\displaystyle\sum_{k = 1}^{\infty}a_k$ сходится.\\
Если же $\displaystyle\sum_{k = 1}^{\infty}a_k$ расходится, то и $\displaystyle\sum_{k = 1}^{\infty}b_k$ расходится.
\end{enumerate}

\subsection{Признак Коши}
Пусть $\textstyle\{a_n\}_{n=1}^{\infty}$ - невозрастающая последовательность, $ a_n \geq 0$. Ряд $\displaystyle\sum_{k = 1}^{\infty}a_k$ сходится тогда и только тогда, когда сходится ряд $\displaystyle\sum_{k = 1}^{\infty}2^k a_{2^k}$
\begin{proof}
Заметим, что $a_2 + 2a_3 + ... + 2^{n-1}a_{2^n} \leq a_2 + a_3 + ... + a_{n^k} \leq 2a_2+ 4a_4 + ... + 2^n a_{2^n}$, тогда из ограниченности частичных сумм ряда $
\displaystyle\sum_{k = 1}^{\infty}2^k a_{2^k}$ следует ограниченность частичных сумм $\displaystyle\sum_{k = 1}^{\infty}a_k$ и наоборот
\end{proof}

\subsection{Сходимость ряда $\displaystyle\sum_{k = 1}^{\infty}\frac{1}{k^p}$}
Ряд $\displaystyle\sum_{k = 1}^{\infty}\frac{1}{k^p}$ сходится при $p > 1$ и расходится при $p \leq 1$
\begin{proof}
При $p < 0$ слагаемое $\frac{1}{k^p}$ не стремится к нулю следовательно ряд расходится.
При $p > 0$: по признаку Коши ряд сходится тогда и только тогда, когда сходится и ряд $\displaystyle\sum_{k = 1}^{\infty}\frac{2^k}{2^{kp}} = \displaystyle\sum_{k = 1}^{\infty}(2^{1-p})^k$, а это сумма геометрической прогрессии, которая сходится при $2^{1-p} < 1$, то есть при $p > 1$ и расходится при $p \leq 1$
\end{proof}

\end{document}