\documentclass[12pt]{article}
\usepackage[utf8]{inputenc}
\usepackage[russian]{babel}
\usepackage[left=1.5cm,right=1.5cm,top=1.5cm,bottom=2cm]{geometry}
\usepackage{graphicx}
\usepackage{color}
\usepackage{titlesec}
\usepackage{amssymb}
\usepackage{mathtools}
\usepackage{minted}
\usepackage{hyperref}
\usepackage{amsmath}
\usepackage{amsfonts}
\usepackage{fancyhdr}
\usepackage{enumitem}
\usepackage{mathbbol}
\usepackage{titlesec}
\usepackage{changepage}
\usepackage{listings}
\usepackage[titles]{tocloft}
\usepackage{tcolorbox}
\usepackage{tikz}
\usepackage{tikz-cd}
\usepackage{ulem}
\usepackage{nicefrac}
\usepackage{indentfirst}
\usepackage{dsfont}
\usepackage{amsthm}

\newtheorem{theorem}{Теорема}%[section]
\newtheorem{lemma}[theorem]{Лемма}
\newtheorem{corollary}[theorem]{Следствие}
\newtheorem{remark}[theorem]{Замечание}
\newtheorem{example}[theorem]{Пример}
\newtheorem{proposition}[theorem]{Предложение}
\theoremstyle{definition}
\newtheorem{definition}[theorem]{Определение}

\usepackage{hyperref}
\hypersetup{
    colorlinks=true,
    linkcolor=cyan,
    filecolor=magenta,      
    urlcolor=blue,
    pdftitle={Overleaf Example},
    pdfpagemode=FullScreen,
    }

\title{Коллоквиум по курсу "Математический анализ", I курс, осенний семестр 2022}
\author{Группа БПМИ2211}

\begin{document}
\maketitle

\section{Предел последовательности, его основные свойства (единственность, арифметические свойства, ограниченность сходящейся последовательности, отделимость).}
\subsection{Предел последовательности}
Если каждому числу $n\in \mathbb{N}$ поставлено в соответствие некоторое число $a_n$, то говорим, что задана {\bf числовая последовательность} $\{a_n\}_{n=1}^\infty$. \\
Говорят, что последовательность $\{a_n\}_{n=1}^\infty$
{\bf сходится} к числу $a$,
если
для каждого числа $\varepsilon>0$
найдется такое натуральное число (номер) $N(\varepsilon)\in \mathbb{N}$,
что $|a_n - a|<\varepsilon$ при каждом $n>N(\varepsilon)$.
То же самое утверждение можно переписать в кванторах $\forall-\exists$
следующим образом:
$$
\forall\varepsilon>0\, \exists N(\varepsilon)\in \mathbb{N}\colon \forall n>N(\varepsilon)\ |a_n-a|<\varepsilon.
$$
Используются обозначения: $\lim\limits_{n\to\infty}a_n=a$ или $a_n\to a$ при $n\to \infty$.

\begin{example}
$1)$ Последоввательность $a_n=\frac{1}{n}$ сходится к числу $a=0$.
Действительно
$$
\Bigl|\frac{1}{n} - 0\Bigr|= \frac{1}{n},
$$
поэтому при $N(\varepsilon)=[\frac{1}{\varepsilon}]+1>\frac{1}{\varepsilon}$ выполнено
$|a_n-a|<\varepsilon$.


$2)$ Последовательность $a_n = (-1)^n$ не имеет предела.
Действительно, если $a$ ее предел, то при достаточно больших $n$
$|a-a_n|<1/2$ и $|a-a_{n+1}|<1/2$, а значит по неравенству треугольника
$2=|a_n - a_{n+1}|<1$, что приводит к противоречию.
\end{example}
\subsection{Основные свойства предела последовательности}
Последовательность $\{a_n\}_{n=1}^\infty$
называется {\bf ограниченной}, если существуют такие числа
$C,c\in \mathbb{R}$, что $c\le a_n\le C$ для каждого $n\in \mathbb{N}$.

\subsubsection{Единственность предела}
Пусть $\lim\limits_{n\to\infty}a_n = a$
и $\lim\limits_{n\to\infty}a_n = b$,
тогда $a=b$.

\begin{proof}
Действительно, если $a\ne b$, то $|a-b|=\varepsilon_0>0$.
Но по определению найдется номер $N_1$,
для которго $|a_n-a|<\frac{\varepsilon_0}{2}$ при $n>N_1$
и найдется номер $N_2$, для которого $|a_n-b|<\frac{\varepsilon_0}{2}$
при $n>N_2$.
Тогда при $n>\max\{N_1, N_2\}$ выполено
$$
\varepsilon_0 = |a-b| = |a - a_n + a_n - b|\le |a-a_n| + |a_n-b|<\varepsilon_0.
$$
Противоречие.
\end{proof}
\subsubsection{Арифметика предела}
Пусть $\lim\limits_{n\to\infty}a_n=a$ и $\lim\limits_{n\to\infty}b_n=b$.
Тогда

$1)\ \lim\limits_{n\to\infty}(\alpha a_n + \beta b_n)=\alpha a+ \beta b$
$\forall \alpha,\beta\in \mathbb{R}$;

$2)\ \lim\limits_{n\to\infty}a_nb_n=ab$;

$3)$ если $b\ne0$, $b_n\ne 0$, то $\lim\limits_{n\to \infty}\frac{a_n}{b_n} = \frac{a}{b}$.

\begin{proof}
Пусть $\varepsilon>0$ --- произвольное число.
Тогда найдется номер $N_1$, для которого $|a_n-a|<\varepsilon$,
и найдется номер $N_2$, для которого $|b_n-b|<\varepsilon$.

$1)$ Получаем, что при $n>N=\max\{N_1, N_2\}$
выполнено
$$
|\alpha a_n +\beta b_n - (\alpha a+\beta b)|
=
|\alpha (a_n-a) +\beta (b_n-b)|
\le
|\alpha| |a_n-a| + |\beta| |b_n-b|< (|\alpha| + |\beta|)\varepsilon.
$$

$2)$ Замечаем, что
$|a_nb_n - ab| = |a_nb_n - ab_n + ab_n - ab|\le
|b_n||a_n - a| + |a||b_n - b|.$
Т.к. сходящаяся последовательность ограничена,
то найдется число $M>0$, для которого
$|b_n|\le M$, поэтому
при $n>N=\max\{N_1, N_2\}$
выполнено
$|a_nb_n - ab|\le (M + |a|)\varepsilon.$

$3)$ Достаточно проверить, что $\frac{1}{b_n}\to \frac{1}{b}$ при $n\to\infty$.
Заметим, что по условию $b\ne 0$, поэтому найдется номер $N_3\in \mathbb{N}$,
для которого, при $n>N_3$, выполнено
$|b_n|>\frac{|b|}{2}$.
Тогда при $n>\max\{N_2, N_3\}$ выполнено
$$
\Bigl|\frac{1}{b_n}-\frac{1}{b}\Bigr| = \frac{|b_n-b|}{|b_n||b|}\le \frac{2}{|b|^2}\cdot \varepsilon.
$$
\end{proof}
\subsubsection{Ограниченность сходящейся последовательности}
Сходящаяся последовательность ограничена.
\begin{proof}
Если $\lim\limits_{n\to\infty}a_n=a$,
то для некоторого $N\in \mathbb{N}$
выполнено
$|a_n-a|<1$ при $n>N$.
Отсюда $|a_n| = |a_n-a+a|\le |a_n-a|+|a|<1+|a|$ при $n>N$.
Значит,
$$
|a_n|\le M=\max\{1+|a|, |a_1|,\ldots, |a_N|\},
$$
т.е. $-M=c\le a_n \le C= M$.
\end{proof}
\subsubsection{Лемма об отделимости}
Если $a_n\to a$ и $a\ne 0$,
то найдется номер $N\in \mathbb{N}$,
для которого $|a_n|>\frac{|a|}{2}>0$
при $n>N$.
\begin{proof}

$\forall \varepsilon > 0: N(\varepsilon) \in \mathbb{N} \ \forall n > N(\varepsilon): |a_n - a| < \varepsilon \\ \text{Возьмем } \varepsilon = \frac{|a|}{2} > 0: \exists N \ \forall n > N: |a_n - a| < \frac{|a|}{2} \\ |a_n| = |a_n + a - a| \geq |a| - |a_n - a| > \frac{|a|}{2}$
\end{proof}
\end{document}
