\documentclass[12pt]{article}
\usepackage[utf8]{inputenc}
\usepackage[russian]{babel}
\usepackage[left=1.5cm,right=1.5cm,top=1.5cm,bottom=2cm]{geometry}
\usepackage{graphicx}
\usepackage{color}
\usepackage{titlesec}
\usepackage{amssymb}
\usepackage{mathtools}
\usepackage{minted}
\usepackage{hyperref}
\usepackage{amsmath}
\usepackage{amsfonts}
\usepackage{fancyhdr}
\usepackage{enumitem}
\usepackage{mathbbol}
\usepackage{titlesec}
\usepackage{changepage}
\usepackage{listings}
\usepackage[titles]{tocloft}
\usepackage{tcolorbox}
\usepackage{tikz}
\usepackage{tikz-cd}
\usepackage{ulem}
\usepackage{nicefrac}
\usepackage{indentfirst}
\usepackage{dsfont}
\usepackage{amsthm}

\newtheorem{theorem}{Теорема}%[section]
\newtheorem{lemma}[theorem]{Лемма}
\newtheorem{corollary}[theorem]{Следствие}
\newtheorem{remark}[theorem]{Замечание}
\newtheorem{example}[theorem]{Пример}
\newtheorem{proposition}[theorem]{Предложение}
\theoremstyle{definition}
\newtheorem{definition}[theorem]{Определение}

\usepackage{hyperref}
\hypersetup{
    colorlinks=true,
    linkcolor=cyan,
    filecolor=magenta,      
    urlcolor=blue,
    pdftitle={Overleaf Example},
    pdfpagemode=FullScreen,
    }

\title{Коллоквиум по курсу "Математический анализ", I курс, осенний семестр 2022}
\author{Группа БПМИ2211}

\begin{document}
\maketitle

\section{Точные верхние и нижние грани, их существование у ограниченных множеств. Теорема Вейерштрасса о пределе монотонной ограниченной последовательности.}
\subsection{Точные верхние и нижние грани, их существование у ограниченных множеств.}
Пусть $A$ --- непустое подмножество вещественных чисел.

Число $b$ называется {\bf верхней гранью} множества $A$,
если $a\le b$ для каждого числа $a\in A$.
Если есть хотя бы одна верхняя грань, то множество называют {\bf ограниченным сверху}.
Наименьшая из верхних граней множества $A$ называется {\bf точной верхней гранью} множества $A$
и обозначается $\sup A$ (супремум).

Число $b$ называется {\bf нижней гранью} множества $A$,
если $b\le a$ для каждого числа $a\in A$.
Если есть хотя бы одна нижняя грань, то множество называют {\bf ограниченным снизу}.
Наибольшая из нижних граней множества $A$ называется {\bf точной нижней гранью} множества $A$
и обозначается $\inf A$ (инфимум).

Ограниченное и сверху и снизу множество называется {\bf ограниченным}.
\begin{example}
{\rm
Пусть $A=(0,1]$.
Тогда $\inf A = 0 \\ \forall x \in A: x \geq 0 \Rightarrow 0$ -- нижняя грань. Если $b$ -- нижняя грань, то $\frac{1}{n} \in A,\ \frac{1}{n} \geq b \Rightarrow 0 := \lim_{n\to\infty}\frac{1}{n} > b$ \\ 
и $\sup A = 1$. 1 -- верхняя грань, т.к. $\forall x \in A: 1 \geq x \\ b$ -- верхняя грань, $b \geq 1, 1\in A$
}\end{example}
Установим существование точных верхних (нижних)
граней у ограниченных сверху (снизу) множеств.

\begin{theorem}
Пусть $A$ --- непустое ограниченное сверху (снизу)
множество. Тогда существует точная верхняя (нижняя)
грань $\sup A$ ($\inf A$).
\end{theorem}

\begin{proof}
Пусть $A$ --- непустое ограниченное сверху множество из условия,
а $B$ --- непустое (по условию) множество его верхних граней.
Тогда $A$ левее $B$ и существует разделяющий $A$ и $B$ элемент $c$.
Он явлется верхней гранью для $A$ и $c\le b$ для каждой верхней грани множества $A$($c$ -- наименьшая из верхних граней).
По определению $c=\sup A$.

Наличие $\inf$ доказывается аналогично или переходом к множеству $-A$.
\end{proof}
\textit{
Отсюда получается полезное утверждение о сходимости монотонной ограниченной последовательности.}
\subsection{Теорема Вейерштрасса о пределе монотонной ограниченной последовательности} 
Пусть последовательность $\{a_n\}_{n=1}^\infty$ не убывает ($a_n\le a_{n+1}$)
и ограничена сверху. Тогда эта последовательность сходится к своему супремуму.

Аналогично, пусть последовательность $\{a_n\}_{n=1}^\infty$ не возрастает ($a_{n+1}\le a_n$)
и ограничена снизу. Тогда эта последовательность сходится к своему инфимуму.

\begin{proof}
Докажем только первое утверждение.
Второе доказывается аналогично или переходом к последовательности $\{-a_n\}_{n=1}^\infty$.

Пусть $M=\sup\{a_n\colon n\in \mathbb{N}\} = \sup\limits_{n\in \mathbb{N}}a_n$.
Тогда для каждого $\varepsilon>0$ найдется номер $N\in \mathbb{N}$, для которого
$M-\varepsilon< a_N$ (иначе $M-\varepsilon$ --- верхняя грань, чего не может быть).
В силу того, что последовательность неубывающая, при каждом $n>N$ выполнено
$$M-\varepsilon< a_N\le a_n\le M< M+\varepsilon.$$
Тем самым, по определению $M=\lim a_n$.
\end{proof}
\textit{В качестве примера см. п.1 билет 6.}
\end{document}
