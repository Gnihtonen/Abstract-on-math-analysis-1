\documentclass{article}
\usepackage[russian]{babel}
\usepackage[a4paper,top=1cm,bottom=2cm,left=3cm,right=3cm,marginparwidth=1.75cm]{geometry}

\usepackage{amsmath}
\usepackage{wrapfig}
\usepackage{amssymb}
\usepackage{graphicx}
\usepackage{mathtools}
\usepackage[export]{adjustbox}
\usepackage[colorlinks=true, allcolors=blue]{hyperref}
\usepackage{hyperref}
\hypersetup{
	colorlinks=true,
	linkcolor=cyan,
	filecolor=magenta,      
	urlcolor=blue,
	pdftitle={Overleaf Example},
	pdfpagemode=FullScreen,
}

\title{}
\author{@lolfucjj - tg}

\DeclareMathOperator*\lowlim{\underline{lim}}
\DeclareMathOperator*\uplim{\overline{lim}}

\begin{document}
	\section*{Вопрос 7}
	\subsection*{Определение подпоследовательности. Её предел(частичный предел последовательности).}
	Пусть дана последовательность $\left\{a_n\right\}_{n=1}^{\infty}$. Так же, пусть задана какая-то возрастаюшая последовательность $\underline{\text{натуральных}}$ чисел $n_1 < n_2 < n_3 < ... $.\\
	Тогда, говорят, что последовательность $b_k = a_{n_{k}}$ является $\textbf{подпоследовательностью}$ последовательности $\left\{a_n\right\}_{n=1}^{\infty}$.\\
	Тогда $\textbf{частичным пределом}$ последовательности  $\left\{a_n\right\}_{n=1}^{\infty}$ называют число $a\in\mathbb{R}$ такое, что $a = \lim\limits_{k\rightarrow\infty}a_{n_{k}}$, для некоторой подпоследовательности $\left\{a_{n_{k}}\right\}_{k=1}^{\infty}$. \\
	То есть другими словами число $a\in\mathbb{R}$ называют  $\textbf{частичным пределом}$, если $a$ является пределом некоторой бесконечной подпоследовательности последовательности  $\left\{a_n\right\}_{n=1}^{\infty}$.
	\subsection*{Предложение №1}
	$\emph{\text{Любая подпоследовательность сходящейся последовательности сходится к пределу}} \\
	\emph{\text{этой последовательности.}}$\\\\
	$\emph{\text{Докозательство}}$. Рассмотрим последовательность $\left\{a_n\right\}_{n=1}^{\infty}$. Пусть $\lim\limits_{n\rightarrow\infty} a_n = A$ и пусть\\ $\left\{a_{n_k}\right\}_{k=1}^{\infty}$ - некоторая подпоследовательность. 
	\\Тогда по поределению предела $\forall (\epsilon > 0)\; \exists N(\epsilon): \forall (n > N(\epsilon))\; |a_n - A| < \epsilon$.\\
	Теперь рассмотрим индексы подпоследовательности. Т.к. $1 \leqslant n_1 $ и $n_{k-1} < n_{k}$ по индукции получим, что $k\leqslant n_k$. Тогда заметим, что для всех $k > N$, получим, что $|a_{n_k} - A| < \epsilon$.
	\subsection*{Верхний и нижний пределы ограниченной последовательности.}
	Рассмотрим последоваетельность $M_n := \sup\limits_{k>n} a_k$ и $m_n := \inf\limits_{k>n}a_k$. Ясно, что посделовательность $M_n$ - невозрастает, а последовательность $m_n$ - неубывает. Поэтому для \underline{\text{ограниченной}} последовательности существует:\[
	\lowlim\limits_{n\rightarrow\infty} a_n := \lim\limits_{n\rightarrow\infty} m_n - \textbf{нижний частичный предел}
	\] \[
	\uplim\limits_{n\rightarrow\infty} a_n := \lim\limits_{n\rightarrow\infty} M_n - \textbf{верхний частичный предел}.
	\]
	\subsection*{Теорема №1}
	\label{subsec:t1}
	\textit{Пусть} $\{a_n\}_{n=1}^{\infty}$ \textit{-- ограниченная последовательность. Тогда} $\uplim\limits_{n \to \infty} a_n, \lowlim\limits_{n \to \infty} a_n$ \textit{-- частичные пределы последовательности} $\{a_n\}_{n=1}^{\infty}$ \textit{и любой другой предел принадлежит отрезку} \[[\lowlim\limits_{n \to \infty} a_n, \uplim\limits_{n \to \infty} a_n]
	\]
	\textit{Доказательство.}
	Покажем, что $M := \uplim\limits_{n \to \infty} a_n$ -- частичный предел. Для этого индуктивно построим последовательность, которая сходится к $\uplim\limits_{n \to \infty} a_n$. Пусть $n_1 = 1$. Пусть индексы $n_1 < n_2 < ... < n_k$ уже построены. Тогда подберём такой номер $n_{k+1} > n_k$, что
	\[
	M_{n_k} - \frac{1}{k+1} < a_{n_{k+1}} \leq M_{n_k}.
	\]
	Как подпоследовательность сходящейся последовательности $M_{n_k} \to M$, поэтому по теореме о сходимости зажатой последовательности (по теореме о двух полицейских и преступнике) получаем, что $\lim\limits_{k \to \infty} a_{n_k} = M$. \\\\
	Аналогично проверяется и то, что $\lowlim\limits_{n \to \infty} a_n$ -- частичный предел. \\\\
	Пусть теперь $a$ -- частичный предел. Это означает, что $a = \lim\limits_{k \to \infty} a_{n_k}$ для некоторой подпоследовательности $\{a_{n_k}\}_{n=1}^{\infty}$. Тогда $m_{n_{k-1}} \leq a_{n_{k}} \leq M_{n_{k-1}}$. По теореме о переходе к пределу в неравенствах получаем, что $\lowlim\limits_{n \to \infty} a_n \leq a \leq \uplim\limits_{n \to \infty} a_n$..
	\subsection*{Следствие из Теоремы 1}
	Теорема Больцано - во всякой ограниченной последовательности можно найти сходящуюся подпоследовательность.
	\subsection*{Теорема №2}
	$\textit{Ограниченная последовательность сходится тогда, и только тогда}, \textit{когда множество её частичных}\\ \textit{пределов  состоит из одного элемента.}$\\\\
	$\textit{Докозательство}$. То, что у сходящейся последовательности только один предел, уже доказано ранее.\\
	Теперь предположим, что у $\underline{\text{ограниченной}}$ последовательности $\left\{a_n\right\}_{n=1}^{\infty}$ только один частичный предел. По доказаному в \hyperref[subsec:t1]{Теорема №1} в частности это означает, что \[
	\lowlim\limits_{n\rightarrow\infty} a_n = \uplim\limits_{n\rightarrow\infty} a_n  = A
	\]
	Тогда, $m_{n-1} \leqslant a_n \leqslant M_{n-1}$, и по теореме о сходимости зажатой последовательности, получаем что $\lim\limits_{n\rightarrow\infty}a_n = A$\textbf{}
\end{document}
