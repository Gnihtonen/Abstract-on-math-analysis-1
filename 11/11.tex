\documentclass[12pt]{article}
\usepackage[utf8]{inputenc}
\usepackage[russian]{babel}
\usepackage[left=1.5cm,right=1.5cm,top=1.5cm,bottom=2cm]{geometry}
\usepackage{graphicx}
\usepackage{color}
\usepackage{titlesec}
\usepackage{amssymb}
\usepackage{mathtools}
\usepackage{minted}
\usepackage{hyperref}
\usepackage{amsmath}
\usepackage{amsfonts}
\usepackage{fancyhdr}
\usepackage{enumitem}
\usepackage{mathbbol}
\usepackage{titlesec}
\usepackage{changepage}
\usepackage{listings}
\usepackage[titles]{tocloft}
\usepackage{tcolorbox}
\usepackage{tikz}
\usepackage{tikz-cd}
\usepackage{ulem}
\usepackage{nicefrac}
\usepackage{indentfirst}
\usepackage{dsfont}
\usepackage{amsthm}

\newtheorem{theorem}{Теорема}%[section]
\newtheorem{lemma}[theorem]{Лемма}
\newtheorem{corollary}[theorem]{Следствие}
\newtheorem{remark}[theorem]{Замечание}
\newtheorem{example}[theorem]{Пример}
\newtheorem{proposition}[theorem]{Предложение}
\theoremstyle{definition}
\newtheorem{definition}[theorem]{Определение}

\usepackage{hyperref}
\hypersetup{
    colorlinks=true,
    linkcolor=cyan,
    filecolor=magenta,      
    urlcolor=blue,
    pdftitle={Overleaf Example},
    pdfpagemode=FullScreen,
    }

\begin{document}

\section{Компакты на $\mathbb{R}$: определение, 3 базовых свойства. Теорема Бореля-Гейне-Лебега о компактности отрезка. Два эквивалентных описания компактных множеств на $\mathbb{R}$.}

\subsection{Определение.}

\begin{definition}
Говорят, что набор множеств $\{U_\alpha\}_{\alpha\in A}$ образует {\bf покрытие}
множества $M\subset \mathbb{R}$, если $M\subset \bigcup\limits_{\alpha\in A}U_\alpha$
(также гооврят, что система $\{U_\alpha\}_{\alpha\in A}$ является покрытием множества $M$).
\end{definition}

\begin{definition}
Множество $K\subset \mathbb{R}$ называется {\bf компактом} (или компактным множеством),
если для каждого покрытия $\{U_\alpha\}_{\alpha\in A}$ множества $K$
открытыми множествами $U_\alpha$ существует конечный поднабор
$\{U_{\alpha_1}, \ldots, U_{\alpha_N}\}$ этих множеств все еще покрывающий $K$
(т.е. $K\subset \bigcup_{j=1}^N U_{\alpha_{j}}$).

Кратко иногда это свойство формулируют так:
Множество $K$ --- компакт, если из каждого покрытия этого множества открытыми множествами
можно выбрать конечное подпокрытие.
\end{definition}

\subsection{Теорема Бореля-Гейне-Лебега о компактности отрезка.}

\begin{theorem}[Борель--Гейне--Лебег]
Каждый отрезок является компактным множеством.
\end{theorem}

\begin{proof}
Предположим, что есть такой отрезок $[a, b]$ и такое его покрытие $\{U_\alpha\}_{\alpha\in A}$
окрытыми множествами, что никакой конечный поднабор этих множеств не покрывает $[a, b]$.
Рассмотрим подотрезки $[a, \frac{a+b}{2}]$ и $[\frac{a+b}{2}, b]$.
Для какой-то из этих половинок никакой конечный поднабор множеств
$\{U_\alpha\}_{\alpha\in A}$ не покрывает эту половинку (если бы для каждой из половинок
был бы покрывающий ее конечный поднабор,
то и весь отрезок бы покрывался объединением этих конечных поднаборов).
Обозначим эту половинку $[a_1, b_1]$. Снова поделим отрезок пополам
и рассмотрим подотрезки $[a_1, \frac{a_1+b_1}{2}]$ и $[\frac{a_1+b_1}{2}, b_1]$.
Для какой-то из этих половинок никакой конечный поднабор множеств
$\{U_\alpha\}_{\alpha\in A}$ не покрывает эту половинку.
Обозначим эту половинку $[a_2, b_2]$.
Продолжая описанную процедуру индуктивно, строим последовательность вложенных отрезков
$[a_{n+1}, b_{n+1}]\subset [a_n, b_n]$ с тем свойством, что
никакой конечный поднабор множеств
$\{U_\alpha\}_{\alpha\in A}$ не покрывает отрезок $[a_n, b_n]$.

Пусть $c\in \bigcap_{n=1}^\infty[a_n, b_n]$. Т.к. $c\in [a, b]$, то для некоторого
индекса $\alpha$ точка $c\in U_\alpha$. Т.к. $U_\alpha$ --- открытое множество,
то найдется такое число $\varepsilon>0$, что $(c-\varepsilon, c+\varepsilon)\subset U_\alpha$.
Т.к. $b_n-a_n = \frac{b-a}{2^n}\to 0$, то $a_n\to c$ и $b_n\to c$.
Тогда для некоторого номера $n_0$ выполнено $a_{n_0}\in (c-\varepsilon, c]$
и $b_{n_0}\in [c, c+\varepsilon)$. Т.е.
$[a_{n_0}, b_{n_0}]\subset (c-\varepsilon, c+\varepsilon)\subset U_\alpha$,
что противоречит построению отрезков $[a_n, b_n]$.
\end{proof}

\subsection{3 базовых свойства компактных множеств.}

\begin{lemma} Пусть $K$ --- компакт. Тогда

$1)\ K$ --- ограниченное множество;

$2)\ K$ --- замкнутое множество;

$3)$ замкнутое подмножество $K$ также компактно.
\end{lemma}

\begin{proof}
$1)$ Заметим, что $K\subset \bigcup_{n=1}^\infty(-n ,n)$. Т.к. $K$ --- компакт,
то у данного покрытия найдется конечное подпокрытие, т.е.
$K\subset \bigcup_{j=1}^m(-n_j ,n_j)$. Пусть $C:=\max\{n_1, \ldots, n_m\}$.
Тогда $K\subset (-C, C)$.

$2)$ Пусть $a\in \mathbb{R}\setminus K$.
Тогда $K\subset \bigcup_{n=1}^\infty U_n$, где
$U_n:=(-\infty, a-\frac{1}{n})\cup (a+\frac{1}{n}, +\infty)$.
Выбрав конечное подпокрытие, получаем, что
$K\subset \bigcup_{j=1}^m U_{n_j}$. Пусть $C:=\max\{n_1, \ldots, n_m\}$.
Тогда $K\subset (-\infty, a-\frac{1}{C})\cup(a+\frac{1}{C}, +\infty)$
и $B_{1/C}(a)\subset \mathbb{R}\setminus K$.

$3)$ Пусть $V\subset K$, $V$ --- замкнутое множество.
Пусть $\{U_\alpha\}_{\alpha\in A}$ --- покрытие множества $V$.
Тогда набор, состоящий из множеств $\{U_\alpha\}_{\alpha\in A}$ и $\mathbb{R}\setminus V$
будет покрытием множества $K$ открытыми множествами.
В нем можно найти конечный поднабор $U_{\alpha_1}, \ldots, U_{\alpha_m}$
и, возможно, $\mathbb{R}\setminus V$, покрывающий множество $K$.
Тогда множество $V$ заведомо покрывается набором $U_{\alpha_1}, \ldots, U_{\alpha_m}$.
\end{proof}

\subsection{Два эквивалентных описания компактных множеств на $\mathbb{R}$.}

\begin{corollary}
Множество $K\subset \mathbb{R}$ компактно тогда и только тогда, когда оно ограничено и замкнуто.
\end{corollary}

\begin{proof}
Компактные множества обязаны быть замкнутыми и ограниченными.
Наоборот, если $K$ ограниченное множество, то $K\subset [-C, C]$ для некоторого числа $C>0$.
Т.к. отрезок --- компактное множество, а $K$ --- замкнутое множество,
то $K$ также будет компактным множеством по предыдущей лемме.
\end{proof}

\begin{corollary}
Множество $K\subset \mathbb{R}$ компактно тогда и только тогда, когда
из каждой последовательности элементов этого множества
можно выбрать подпоследовательность, сходящуюся к элементу этого множества.
\end{corollary}

\begin{proof}
Если множество $K$ --- компактно, то оно замкнуто и ограничено.
Пусть $\{a_n\}_{n=1}^\infty\subset K$. По теореме Больцано в данной последовательности
найдется сходящаяся подпоследовательность $a_{n_k}\to a$. В силу замкнутости множества $K$
получаем, что $a\in K$.

Наоборот, пусть из каждой последовательности элементов множества $K$
можно выбрать подпоследовательность, сходящуюся к элементу этого множества.
Если бы множество $K$ не являлось ограниченным, то для каждого $n\in \mathbb{N}$
была бы точка $a_n\in K$, $|a_n|>n$. Из такой последовательности невозможно выбрать
сходящуюся подпоследовательность. Пусть теперь $a_n\in K$, $a_n\to a$.
По условию, из этой последовательности можно выбрать подпоследовательность $a_{n_k}$,
сходящуюся к точке множества $K$, т.е. $a_{n_k}\to b\in K$. В силу единственности предела
и совпадения предела подпоследовательности с пределом всей последовательности получаем,
что $a=b\in K$.
\end{proof}

\end{document}
