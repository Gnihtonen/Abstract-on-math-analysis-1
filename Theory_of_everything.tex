\documentclass[12pt]{article}
\usepackage[utf8]{inputenc}
\usepackage[russian]{babel}
\usepackage[left=1.5cm,right=1.5cm,top=1.5cm,bottom=2cm]{geometry}
\usepackage{graphicx}
\usepackage{color}
\usepackage{titlesec}
\usepackage{amssymb}
\usepackage{mathtools}
\usepackage{minted}
\usepackage{hyperref}
\usepackage{amsmath}
\usepackage{amsfonts}
\usepackage{fancyhdr}
\usepackage{mathdots}
\usepackage{enumitem}
\usepackage{mathbbol}
\usepackage{titlesec}
\usepackage{changepage}
\usepackage{listings}
\usepackage[titles]{tocloft}
\usepackage{tcolorbox}
\usepackage{tikz}
\usepackage{tikz-cd}
\usepackage{ulem}
\usepackage{nicefrac}
\usepackage{indentfirst}
\usepackage{dsfont}
\usepackage{amsthm}

\newtheorem{theorem}{Теорема}%[section]
\newtheorem{lemma}[theorem]{Лемма}
\newtheorem{corollary}[theorem]{Следствие}
\newtheorem{remark}[theorem]{Замечание}
\newtheorem{example}[theorem]{Пример}
\newtheorem{proposition}[theorem]{Предложение}
\theoremstyle{definition}
\newtheorem{definition}[theorem]{Определение}
\newtheorem{statement}[]{}

\usepackage{hyperref}

\DeclareRobustCommand{\divby}{%
  \mathrel{\vbox{\baselineskip.65ex\lineskiplimit0pt\hbox{.}\hbox{.}\hbox{.}}}%
}

\DeclareMathOperator*\lowlim{\underline{lim}}
\DeclareMathOperator*\uplim{\overline{lim}}

\newcommand\nn{$\mathds{N}$}

\title{Коллоквиум по курсу "Математический анализ", I курс, осенний семестр 2022}
\author{Группа БПМИ2211}

\begin{document}
\maketitle

\section{Рациональные и вещественные числа. Десятичные дроби. Принцип полноты, его выполнение для десятичных дробей.}
\subsubsection*{Рациональные числа}
\textbf{Рациональные числа} – это числа вида $\frac{p}{q}$, где p – целое число, q – натуральное число, причём два числа $\frac{p_1}{q_1}$  и $\frac{p_2}{q_2}$  считаются равными, если $p_1q_2=p_2q_1$. Все свойства натуральных, целых, рациональных чисел и операций над ними будем считать известными.

\subsection*{Десятичные дроби и вещественные числа}

Каждое рациональное число можно представить в виде конечной или бесконечной периодической десятичной дроби, например: $\frac{1}{10}=0.1,   \frac{1}{7}=0.(142857)$. Пусть $0.(9)=x$, тогда $10x=9+x$, значит, $0,(9)=1$, поэтому десятичные записи с периодом 9 рассматривать не будем.

Множество \textbf{вещественных (действительных)} чисел отождествляется с множеством всех десятичных дробей вида $ \pm a_0.a_1 a_2\ldots$,  где $a_0 \in \mathbb {N} \cup \{0\}$, $a_j \in \{0,…,9\}$, и записи, в которых с какого-то момента стоят одни девятки, запрещены. Число $\pm 0.00\ldots$ совпадает с числом 0 и называется нулём. Ненулевое число называется положительным, если в его записи стоит знак + (который обычно опускается). Ненулевое число называется отрицательным, если в его записи стоит знак $-$. B вещественные числа естественным образом вложены рациональные.

На множестве вещественных чисел также определены операции сложения и умножения, для которых справедливы все их естественные свойства (множество вещественных чисел является полем).

На вещественных числах задано \textbf{отношение порядка} следующим образом: на положительных вещественных числах задан лексикографический порядок, т. е. $a_0.a_1a_2\ldots\leq b_0.b_1 b_2\ldots$ тогда и только тогда, когда $a_0.a_1 a_2 \ldots = b_0.b_1 b_2 \ldots $ или найдётся разряд k, для которого $a_0=b_0,\ldots,a_{k-1}=b_{k-1}$ и $a_k<b_k$, который естественным образом переносится на отрицательные.

Для вещественных чисел определён модуль числа $|a|$, равный $-a$ при $a < 0$ и $a$ при $a \geq 0$. Напомним, что для модуля выполнено \textbf{неравенство треугольника} $|a+b| \leq |a|+|b|$. Из неравенства треугольника следует, что $||a|-|b|| \leq |a+b|$.

\subsection*{Принцип полноты}
Будем говорить, что множество чисел A лежит \textbf{левее} множества B, если для каждого $a \in A$ и каждого $b \in B$ выполняется неравенство $a \leq b$. Например, если $A=\{a \in \mathbb{Q}:a<4\}$, $B=\{b \in \mathbb{Q}:b>4\}$, то A левее B.

Если множество A левее множества B, то говорят, что число $c$ \textbf{разделяет} множества A и B, если $a \leq c$ для каждого $a \in A$ и $c \leq b$ для каждого $b \in B$. Например, число 4 разделяет множества A и B, заданные выше.

Будем говорить, что на множестве чисел выполнен \textbf{принцип полноты}, если для произвольных непустых подмножеств A левее B нашего множества найдётся разделяющий их элемент.


\textbf{Теорема.} \textit{На множестве вещественных чисел выполняется принцип полноты.}

\begin{proof} Пусть A и B – непустые множества чисел, причём A левее B. Если A состоит только из неположительных чисел, а B – только из неотрицательных, то нуль разделяет множества A и B.

Предположим, что в A есть положительный элемент, тогда B состоит только из положительных чисел (случай, когда в B есть отрицательное число, рассматривается аналогично). Построим число $c=c_0.c_1 c_2\ldots$, разделяющее A и B.

Рассмотрим множество всех целых неотрицательных чисел, с которых начинаются элементы множества B (это множество состоит из целых неотрицательных чисел в силу того, что в B есть только положительные числа). Пусть $b_0$ – наименьшее из таких чисел и положим $c_0=b_0$. Теперь рассмотрим все числа в множестве B, начинающиеся с $c_0$, и найдём у них наименьшую первую цифру после запятой. Пусть эта цифра $b_1$, тогда полагаем $c_1=b_1$. Теперь рассмотрим все числа в множестве B, начинающиеся с $c_0.c_1$, и найдём у них наименьшую вторую цифру после запятой. Пусть эта цифра $b_2$, тогда полагаем $c_2=b_2$. Аналогично ищутся остальные цифры числа $c$.

Таким образом построена бесконечная десятичная дробь $c_0.c_1 c_2\ldots$. Заметим, что если бы у построенной десятичной записи с какого-то момента шли бы только девятки, то и в B было бы число, в записи которого с какого-то момента участвуют только девятки, но такие записи мы запретили.

Покажем, что построенное число разделяет множества A и B.

Во-первых, по построению $c \leq b$ для каждого $b \in B$. Действительно, либо $b=c$ (тогда всё ОК), либо $b \neq c$. Во втором случае пусть $b_0=c_0,\ldots, b_{k-1}=c_{k-1}$ и $b_k \neq c_k$. Тогда, по построению числа $c$, $c_k<b_k$ и $c<b$.

Покажем, что $a \leq c$ для каждого $a \in A$. Предположим, что $a>c$, т. е. $a \geq c$ и $a\neq c$. Тогда найдётся позиция k, для которой $a_0=c_0,\ldots, a_{k-1}=c_{k-1}$ и $a_k>c_k$. Но по построению числа $c$ есть такой $b\in B$, что $b_0=c_0,\ldots b_k=c_k$, а значит $a>b$, что противоречит условию A левее B.
\end{proof}

\section{Иррациональность числа $\sqrt{2}$ (т.е. положительного решения уравнения $x^2 = 2$), его существование в рамках вещественных чисел, как следствие принципа полноты.}

\subsection{Иррациональность числа $\sqrt{2}$ (т.е. положительного решения уравнения $x^2 = 2$).}

Докажем, что рациональных решений уравнения $x^2 = 2$ не существует.  (от противного)  
\begin{proof}
Предположим, что $\frac{p}{q}$ – такое решение, где $p \in \mathbb{Z}$, $q\in\mathbb{N}$ и дробь несократима, т.е. нет общ делителей.
Тогда $2 = \frac{p^2}{q^2} \Leftrightarrow 2q^2=p^2 \Rightarrow p^2\divby2\Rightarrow p\divby2\Rightarrow p = 2p_1 \Rightarrow 2q^2 = 4p_1^2 \Leftrightarrow \newline \Leftrightarrow q^2 = 2p_1^2 \Rightarrow q \divby2 \Rightarrow $ $p$ и $q$ - чётные, а $\frac{p}{q} - $ сократимая дробь $\Rightarrow$ противоречие.
Таким образом, доказали, что $\sqrt{2} \not\in \mathbb{Q}$.
\end{proof}

\subsection{Существование $\sqrt{2}$ в рамках вещественных чисел.}

	Объясним чем с точки зрения структуры множества чисел обусловлено такое “отсутствие” $\sqrt{2}$.
Пусть $A = \{a \in \mathbb{R}: a > 0, a^2 \leq 2\}$ и $B = \{b\in\mathbb{R}: b > 0, b^2 \geq 2\}$. Заметим, что множество $A$ лежит левее множества $B$, так как $0 < b^2 - a^2 = (b-a)\cdot(b+a)$ для каждых $a \in A$ и $b \in B$, и $a + b > 0$. Если бы существовало число $c$, разделяющее $A$ и $B$, то обязательно $c^2 = 2$. 

Действительно, во-первых, заметим, что $1 \leq c \leq 2$ т.к. $1 \in A, 2 \in B$.

Теперь, если $c^2 < 2$, то число $c + \frac{2-c^2}{5} \in A$, т.к. 
$(c+\frac{2-c^2}{5})^2 = c^2 + 2c \cdot \frac{2-c^2}{5} + (\frac{2-c^2}{5})^2 \leq c^2 + 4 \cdot\frac{2-c^2}{5} + \frac{2-c^2}{5}  \leq 2$, но $c + \frac{2-c^2}{5} > c \Rightarrow c$ не разделяет $A$ и $B$.

Если  $c^2 > 2$, то число $c - \frac{c^2-2}{4} \in B$, т.к.
$(c-\frac{c^2-2}{4})^2 \geq c^2 - 2c \cdot \frac{c^2-2}{4} \geq c^2 - 4 \cdot\frac{c^2-2}{4} = 2$, но $c - \frac{c^2-2}{4} < c \Rightarrow c$ не разделяет $A$ и $B$.

Таким образом, $c^2 = 2$. (Так как $c^2 = 2$, где $c$ разделяет $A$ и $B$, то из принципа полноты для десятичных дробей следует, что число $c$ существует.)

\section{Предел последовательности, его основные свойства (единственность, арифметические свойства, ограниченность сходящейся последовательности, отделимость).}
\subsection{Предел последовательности}
Если каждому числу $n\in \mathbb{N}$ поставлено в соответствие некоторое число $a_n$, то говорим, что задана {\bf числовая последовательность} $\{a_n\}_{n=1}^\infty$. \\
Говорят, что последовательность $\{a_n\}_{n=1}^\infty$
{\bf сходится} к числу $a$,
если
для каждого числа $\varepsilon>0$
найдется такое натуральное число (номер) $N(\varepsilon)\in \mathbb{N}$,
что $|a_n - a|<\varepsilon$ при каждом $n>N(\varepsilon)$.
То же самое утверждение можно переписать в кванторах $\forall-\exists$
следующим образом:
$$
\forall\varepsilon>0\, \exists N(\varepsilon)\in \mathbb{N}\colon \forall n>N(\varepsilon)\ |a_n-a|<\varepsilon.
$$
Используются обозначения: $\lim\limits_{n\to\infty}a_n=a$ или $a_n\to a$ при $n\to \infty$.

\begin{example}
$1)$ Последоввательность $a_n=\frac{1}{n}$ сходится к числу $a=0$.
Действительно
$$
\Bigl|\frac{1}{n} - 0\Bigr|= \frac{1}{n},
$$
поэтому при $N(\varepsilon)=[\frac{1}{\varepsilon}]+1>\frac{1}{\varepsilon}$ выполнено
$|a_n-a|<\varepsilon$.


$2)$ Последовательность $a_n = (-1)^n$ не имеет предела.
Действительно, если $a$ ее предел, то при достаточно больших $n$
$|a-a_n|<1/2$ и $|a-a_{n+1}|<1/2$, а значит по неравенству треугольника
$2=|a_n - a_{n+1}|<1$, что приводит к противоречию.
\end{example}
\subsection{Основные свойства предела последовательности}
Последовательность $\{a_n\}_{n=1}^\infty$
называется {\bf ограниченной}, если существуют такие числа
$C,c\in \mathbb{R}$, что $c\le a_n\le C$ для каждого $n\in \mathbb{N}$.

\subsubsection{Единственность предела}
Пусть $\lim\limits_{n\to\infty}a_n = a$
и $\lim\limits_{n\to\infty}a_n = b$,
тогда $a=b$.

\begin{proof}
Действительно, если $a\ne b$, то $|a-b|=\varepsilon_0>0$.
Но по определению найдется номер $N_1$,
для которго $|a_n-a|<\frac{\varepsilon_0}{2}$ при $n>N_1$
и найдется номер $N_2$, для которого $|a_n-b|<\frac{\varepsilon_0}{2}$
при $n>N_2$.
Тогда при $n>\max\{N_1, N_2\}$ выполено
$$
\varepsilon_0 = |a-b| = |a - a_n + a_n - b|\le |a-a_n| + |a_n-b|<\varepsilon_0.
$$
Противоречие.
\end{proof}
\subsubsection{Арифметика предела}
Пусть $\lim\limits_{n\to\infty}a_n=a$ и $\lim\limits_{n\to\infty}b_n=b$.
Тогда

$1)\ \lim\limits_{n\to\infty}(\alpha a_n + \beta b_n)=\alpha a+ \beta b$
$\forall \alpha,\beta\in \mathbb{R}$;

$2)\ \lim\limits_{n\to\infty}a_nb_n=ab$;

$3)$ если $b\ne0$, $b_n\ne 0$, то $\lim\limits_{n\to \infty}\frac{a_n}{b_n} = \frac{a}{b}$.

\begin{proof}
Пусть $\varepsilon>0$ --- произвольное число.
Тогда найдется номер $N_1$, для которого $|a_n-a|<\varepsilon$,
и найдется номер $N_2$, для которого $|b_n-b|<\varepsilon$.

$1)$ Получаем, что при $n>N=\max\{N_1, N_2\}$
выполнено
$$
|\alpha a_n +\beta b_n - (\alpha a+\beta b)|
=
|\alpha (a_n-a) +\beta (b_n-b)|
\le
|\alpha| |a_n-a| + |\beta| |b_n-b|< (|\alpha| + |\beta|)\varepsilon.
$$

$2)$ Замечаем, что
$|a_nb_n - ab| = |a_nb_n - ab_n + ab_n - ab|\le
|b_n||a_n - a| + |a||b_n - b|.$
Т.к. сходящаяся последовательность ограничена,
то найдется число $M>0$, для которого
$|b_n|\le M$, поэтому
при $n>N=\max\{N_1, N_2\}$
выполнено
$|a_nb_n - ab|\le (M + |a|)\varepsilon.$

$3)$ Достаточно проверить, что $\frac{1}{b_n}\to \frac{1}{b}$ при $n\to\infty$.
Заметим, что по условию $b\ne 0$, поэтому найдется номер $N_3\in \mathbb{N}$,
для которого, при $n>N_3$, выполнено
$|b_n|>\frac{|b|}{2}$.
Тогда при $n>\max\{N_2, N_3\}$ выполнено
$$
\Bigl|\frac{1}{b_n}-\frac{1}{b}\Bigr| = \frac{|b_n-b|}{|b_n||b|}\le \frac{2}{|b|^2}\cdot \varepsilon.
$$
\end{proof}
\subsubsection{Ограниченность сходящейся последовательности}
Сходящаяся последовательность ограничена.
\begin{proof}
Если $\lim\limits_{n\to\infty}a_n=a$,
то для некоторого $N\in \mathbb{N}$
выполнено
$|a_n-a|<1$ при $n>N$.
Отсюда $|a_n| = |a_n-a+a|\le |a_n-a|+|a|<1+|a|$ при $n>N$.
Значит,
$$
|a_n|\le M=\max\{1+|a|, |a_1|,\ldots, |a_N|\},
$$
т.е. $-M=c\le a_n \le C= M$.
\end{proof}
\subsubsection{Лемма об отделимости}
Если $a_n\to a$ и $a\ne 0$,
то найдется номер $N\in \mathbb{N}$,
для которого $|a_n|>\frac{|a|}{2}>0$
при $n>N$.
\begin{proof}

$\forall \varepsilon > 0: N(\varepsilon) \in \mathbb{N} \ \forall n > N(\varepsilon): |a_n - a| < \varepsilon \\ \text{Возьмем } \varepsilon = \frac{|a|}{2} > 0: \exists N \ \forall n > N: |a_n - a| < \frac{|a|}{2} \\ |a_n| = |a_n + a - a| \geq |a| - |a_n - a| > \frac{|a|}{2}$
\end{proof}

\section{Переход к пределу в неравенствах. Принцип вложенных отрезков и геометрическая интерпретация вещественных чисел, вещественная прямая.}
\subsection{Переход к пределу в неравенствах}
% \begin{statement}
Пусть $\lim\limits_{n\to\infty} a_n = a,\;\lim\limits_{n\to\infty} b_n = b$, тогда $\exists\,N\;\forall n > N: a_n \le b_n \Rightarrow a \le b$.
% \end{statement}
\begin{proof}
Предположим $a - b = \varepsilon_0 > 0 \Rightarrow\\\Rightarrow \exists N_1,\, N_2: \left|a_n - a\right| < \frac{\varepsilon_0}{2}\;\forall\, n>N_1,\; \left|b_n - b\right| < \frac{\varepsilon_0}{2}\; \forall\, n>N_2 \Rightarrow\\\Rightarrow \epsilon_0 = a - b = a - a_n + a_n - b + b_n - b_n \le a - a_n + b_n - b < \varepsilon_0$ -- противоречие. 
\end{proof}
\subsection{Лемма о зажатой последовательности}
\begin{lemma}
Пусть $\lim\limits_{n\to\infty} a_n = \lim\limits_{n\to\infty} b_n = a$. Тогда $\exists N\;\forall n>N: a_n \le c_n \le b_n \Rightarrow \lim\limits_{n\to\infty} c_n = a$.
\end{lemma}
\begin{proof}
По определению $\forall\varepsilon\;\exists\, N_1\in\mathbb{N},\,N_2\in\mathbb{N}: \left|a - a_n\right| < \varepsilon\,\forall n > N_1,\; \left|b - b_m\right| < \varepsilon\, \forall m > N_2 \Rightarrow \forall k > \max\{N, N_1, N_2\}: a - \varepsilon < a_k \le c_k \le b_k < a + \varepsilon \Rightarrow \lim\limits_{n\to\infty} c_n = a$
\end{proof}
\subsection{Вещественная прямая}
Пусть $a, b \in\mathbb{R}$ и $a < b$. Множества $[a;\, b] := \{x\in\mathbb{R}: a \le x \le b\},\; (a;\, b):=\{x\in\mathbb{R}: a < x < b\}$ называются отрезком и интервалом соответственно.\\
Длина отрезка (интервала) -- величина $b - a$. 
\subsection{Принцип вложенных отрезков}
\begin{theorem}
Всякая последовательность $\{[a_n;\, b_n]\}_{n = 1}^\infty$ вложенных отрезков (то есть таких, что $[a_{n + 1};\, b_{n + 1}] \subset [a_n;\, b_n]$) имеет общую точку. Кроме того, если длины отрезков стремятся к нулю, то есть $b_n - a_n \to 0$, то такая общая точка только одна.
\end{theorem}
\begin{proof}
По условию $[a_{n + 1};\, b_{n + 1}] \subset [a_n;\, b_n]$, откуда $a_n \le a_{n + 1} \le b_{n + 1} \le b_n$.\\
Пусть $n < m$, тогда $a_n \le a_m \le b_m \Rightarrow a_n < b_m$. При $n > m$ получим, что $a_n \le b_n \le b_m \Rightarrow a_n < b_m$. Таким образом, $a_n < b_m\;\forall\;n,m\in\mathbb{N}$, тогда если $A:=\{a_n,\,n\in\mathbb{N}\},\,B:=\{b_m,\,m\in\mathbb{N}\}$, то $A$ левее $B$.\\
Тогда по принципу полноты $\exists\;c\in\mathbb{R}:\; a_n \le c \le b_m\;\forall\;n, m\in\mathbb{N}$.\\
В частности, $a_n \le c \le  b_n \Rightarrow c\in[a_n;\, b_n]$.\\
\\
Пусть общих точек две: $c$ и $c'$. Без ограничения общности, скажем, что $c < c'$. \\
Тогда, получим, что $a_n \le c \le c' \le b_n$ и $c' - c \le b_n - a_n$. \\
Но $\lim\limits_{n\to\infty} b_n - a_n = 0 \Rightarrow \forall\varepsilon>0\;\exists N(\varepsilon)\in\mathbb{N}:\forall n\ge N(\varepsilon) \left|0 - b_n + a_n\right| < \varepsilon$. \\
Пусть $\varepsilon = c' - c$, тогда $\left|a_n - b_n\right| < c' - c \Rightarrow b_n - a_n < c' - c$ -- противоречие. 
\end{proof}
\subsection{Геометрическая интерпретация вещественных чисел}
Сопоставим десятичной дроби $0.a_1a_2...$ последовательность вложенных отрезков по следующему правилу. \\
Разделим отрезок $[0;\, 1]$ на 10 равных частей и выберем из получившихся частей $a_1+1$-ый по счету.\\
Проделываем ту же самую процедуру с выбранным отрезком и выбираем $a_2+1$-ый по счету. И так далее.\\
Получаем последовательность вложенных отрезков. Причем длина отрезка, получаемого на $n$-ом шаге, равна $\frac{1}{10^n}$. \\
По теореме 1 существует единственная ($\lim\limits_{n\to\infty} \frac{1}{10^n} = 0$) общая точка получившейся последовательности вложенных отрезков, которая совпадает с $0.a_1a_2$.

\section{Точные верхние и нижние грани, их существование у ограниченных множеств. Теорема Вейерштрасса о пределе монотонной ограниченной последовательности.}
\subsection{Точные верхние и нижние грани, их существование у ограниченных множеств.}
Пусть $A$ --- непустое подмножество вещественных чисел.

Число $b$ называется {\bf верхней гранью} множества $A$,
если $a\le b$ для каждого числа $a\in A$.
Если есть хотя бы одна верхняя грань, то множество называют {\bf ограниченным сверху}.
Наименьшая из верхних граней множества $A$ называется {\bf точной верхней гранью} множества $A$
и обозначается $\sup A$ (супремум).

Число $b$ называется {\bf нижней гранью} множества $A$,
если $b\le a$ для каждого числа $a\in A$.
Если есть хотя бы одна нижняя грань, то множество называют {\bf ограниченным снизу}.
Наибольшая из нижних граней множества $A$ называется {\bf точной нижней гранью} множества $A$
и обозначается $\inf A$ (инфимум).

Ограниченное и сверху и снизу множество называется {\bf ограниченным}.
\begin{example}
{\rm
Пусть $A=(0,1]$.
Тогда $\inf A = 0 \\ \forall x \in A: x \geq 0 \Rightarrow 0$ -- нижняя грань. Если $b$ -- нижняя грань, то $\frac{1}{n} \in A,\ \frac{1}{n} \geq b \Rightarrow 0 := \lim_{n\to\infty}\frac{1}{n} > b$ \\ 
и $\sup A = 1$. 1 -- верхняя грань, т.к. $\forall x \in A: 1 \geq x \\ b$ -- верхняя грань, $b \geq 1, 1\in A$
}\end{example}
Установим существование точных верхних (нижних)
граней у ограниченных сверху (снизу) множеств.

\begin{theorem}
Пусть $A$ --- непустое ограниченное сверху (снизу)
множество. Тогда существует точная верхняя (нижняя)
грань $\sup A$ ($\inf A$).
\end{theorem}

\begin{proof}
Пусть $A$ --- непустое ограниченное сверху множество из условия,
а $B$ --- непустое (по условию) множество его верхних граней.
Тогда $A$ левее $B$ и существует разделяющий $A$ и $B$ элемент $c$.
Он явлется верхней гранью для $A$ и $c\le b$ для каждой верхней грани множества $A$($c$ -- наименьшая из верхних граней).
По определению $c=\sup A$.

Наличие $\inf$ доказывается аналогично или переходом к множеству $-A$.
\end{proof}
\textit{
Отсюда получается полезное утверждение о сходимости монотонной ограниченной последовательности.}
\subsection{Теорема Вейерштрасса о пределе монотонной ограниченной последовательности} 
Пусть последовательность $\{a_n\}_{n=1}^\infty$ не убывает ($a_n\le a_{n+1}$)
и ограничена сверху. Тогда эта последовательность сходится к своему супремуму.

Аналогично, пусть последовательность $\{a_n\}_{n=1}^\infty$ не возрастает ($a_{n+1}\le a_n$)
и ограничена снизу. Тогда эта последовательность сходится к своему инфимуму.

\begin{proof}
Докажем только первое утверждение.
Второе доказывается аналогично или переходом к последовательности $\{-a_n\}_{n=1}^\infty$.

Пусть $M=\sup\{a_n\colon n\in \mathbb{N}\} = \sup\limits_{n\in \mathbb{N}}a_n$.
Тогда для каждого $\varepsilon>0$ найдется номер $N\in \mathbb{N}$, для которого
$M-\varepsilon< a_N$ (иначе $M-\varepsilon$ --- верхняя грань, чего не может быть).
В силу того, что последовательность неубывающая, при каждом $n>N$ выполнено
$$M-\varepsilon< a_N\le a_n\le M< M+\varepsilon.$$
Тем самым, по определению $M=\lim a_n$.
\end{proof}
\textit{В качестве примера см. п.1 билет 6.}

\section{Вычисление $\sqrt{2}$ с помощью рекурентной формулы $a_{n+1}=\frac{1}{2}\bigl(a_n + \frac{2}{a_n}\bigr), a_1 = 2$, обоснование сходимости и оценка скорости сходимости. Число $e$ (определение и обоснование корректности).}
\subsection{Вычисление $\sqrt{2}$ с помощью рекурентной формулы $a_{n+1}=\frac{1}{2}\bigl(a_n + \frac{2}{a_n}\bigr)$}
{\rm 
$(\sqrt{a} - \sqrt{b})^2 \geq 0, \  a - 2\sqrt{ab} + b \geq 0, \ a + b \geq 2\sqrt{ab}$ \\ 
Заметим, что 
$$
a_{n+1}=\frac{1}{2}\Bigl(a_n + \frac{2}{a_n}\Bigr)\ge \frac{1}{2}\cdot 2\sqrt{a_n\cdot\frac{2}{a_n}}=\sqrt2.
$$
Поэтому $a_n\ge\sqrt 2$.
Кроме того
$
a_{n+1}=\frac{1}{2}\bigl(a_n + \frac{2}{a_n}\bigr) \le \frac{1}{2}\bigl(a_n + \frac{a_n^2}{a_n}\bigr)=a_n.
$ \\ 
$\{a_n\}_{n=1}^\infty$ ограничена снизу и не возрастает, тогда по т. Вейерштрасса у последовательности $\{a_n\}_{n=1}^\infty$
существует предел $a$. Т.к. $a_n\ge \sqrt2>0$, то и $a> 0$.
Тогда, по арифметике предела получаем
$a =\frac{1}{2}\bigl(a + \frac{2}{a}\bigr)$, откуда $a=\sqrt2$.

Исследуем теперь скорость сходимости:
$$
|a_{n+1} - \sqrt2| = \frac{|a_n^2 - 2 a_n\sqrt2 +2|}{2a_n}
=
\frac{(a_n - \sqrt2)^2}{2a_n}
\le
\frac{(a_n - \sqrt2)^2}{2\sqrt2}\le (a_n - \sqrt2)^2.
$$
Индуктивно получаем
$$
|a_{n+1} - \sqrt2| \le (a_n - \sqrt2)^2 \le (a_{n-1} - \sqrt2)^4
\le (a_{n-2} - \sqrt2)^8\le (a_1 - \sqrt2)^{2^{n}} = (2 - \sqrt2)^{2^{n}}.
$$
Заметим, что $q:=2-\sqrt2 <1$, поэтому полученная скорость сходимость $q^{2^n}$
быстрее экспоненциальной $q^n$
(в смысле количества применений рекуррентной формулы для достижения заданной точности).
}
\subsection{Число $e$}
У последовательности $a_n = \left(1+\frac{1}{n}\right)^n$ есть предел, который называют \textbf{числом} $e$. \\ 
Пусть $a_n = \Bigl(1+\frac{1}{n}\Bigr)^{n}$.
По биному Ньютона
$$
a_n = \sum_{k=0}^{n}C_n^k \frac{1}{n^k}
= 2+\sum_{k=2}^{n}\frac{1}{k!}\frac{n\cdot (n-1)\cdot\ldots\cdot (n-k+1)}{n^k}
=2+\sum_{k=2}^{n}\frac{1}{k!}\Bigl(1-\frac{1}{n}\Bigr)\cdot\ldots\cdot\Bigl(1-\frac{k-1}{n}\Bigr).
$$
Отсюда, во-первых, получаем, что
$$
a_n\le2+\sum_{k=2}^{n}\frac{1}{k!}\le 2+\sum_{k=2}^{n}\frac{1}{2^{k-1}}< 3,
$$
где было использовано неравенство $k! = 1\cdot2\cdot3\cdot\ldots\cdot k \ge 2^{k-1}$
при $k\ge 2$.
Во-вторых,
\begin{multline*}
a_n=2+\sum_{k=2}^{n}\frac{1}{k!}\Bigl(1-\frac{1}{n}\Bigr)\cdot\ldots\cdot\Bigl(1-\frac{k-1}{n}\Bigr)
\le
2+\sum_{k=2}^{n}\frac{1}{k!}\Bigl(1-\frac{1}{n+1}\Bigr)\cdot\ldots\cdot\Bigl(1-\frac{k-1}{n+1}\Bigr)
\\
\le
2+\sum_{k=2}^{n+1}\frac{1}{k!}\Bigl(1-\frac{1}{n+1}\Bigr)\cdot\ldots\cdot\Bigl(1-\frac{k-1}{n+1}\Bigr)=a_{n+1}.
\end{multline*}
Таким образом, последовательность $a_n$ --- неубывает и ограничена сверху, а значит имеет предел,
который называют {\bf числом $e$}.

\section{Подпоследовательность и частичные пределы. Верхний и нижний пределы ограниченной последовательности, их связь с множеством частичных пределов этой последовательности. Критерий сходимости последовательности в терминах частичных пределов.}
	\subsection{Определение подпоследовательности. Её предел(частичный предел последовательности).}
	Пусть дана последовательность $\left\{a_n\right\}_{n=1}^{\infty}$. Так же, пусть задана какая-то возрастаюшая последовательность $\underline{\text{натуральных}}$ чисел $n_1 < n_2 < n_3 < ... $.\\
	Тогда, говорят, что последовательность $b_k = a_{n_{k}}$ является $\textbf{подпоследовательностью}$ последовательности $\left\{a_n\right\}_{n=1}^{\infty}$.\\
	Тогда $\textbf{частичным пределом}$ последовательности  $\left\{a_n\right\}_{n=1}^{\infty}$ называют число $a\in\mathbb{R}$ такое, что $a = \lim\limits_{k\rightarrow\infty}a_{n_{k}}$, для некоторой подпоследовательности $\left\{a_{n_{k}}\right\}_{k=1}^{\infty}$. \\
	То есть другими словами число $a\in\mathbb{R}$ называют  $\textbf{частичным пределом}$, если $a$ является пределом некоторой бесконечной подпоследовательности последовательности  $\left\{a_n\right\}_{n=1}^{\infty}$.
	\subsection{Предложение №1}
	$\emph{\text{Любая подпоследовательность сходящейся последовательности сходится к пределу}} \\
	\emph{\text{этой последовательности.}}$\\\\
	$\emph{\text{Докозательство}}$. Рассмотрим последовательность $\left\{a_n\right\}_{n=1}^{\infty}$. Пусть $\lim\limits_{n\rightarrow\infty} a_n = A$ и пусть\\ $\left\{a_{n_k}\right\}_{k=1}^{\infty}$ - некоторая подпоследовательность. 
	\\Тогда по поределению предела $\forall (\epsilon > 0)\; \exists N(\epsilon): \forall (n > N(\epsilon))\; |a_n - A| < \epsilon$.\\
	Теперь рассмотрим индексы подпоследовательности. Т.к. $1 \leqslant n_1 $ и $n_{k-1} < n_{k}$ по индукции получим, что $k\leqslant n_k$. Тогда заметим, что для всех $k > N$, получим, что $|a_{n_k} - A| < \epsilon$.
	\subsection{Верхний и нижний пределы ограниченной последовательности.}
	Рассмотрим последоваетельность $M_n := \sup\limits_{k>n} a_k$ и $m_n := \inf\limits_{k>n}a_k$. Ясно, что посделовательность $M_n$ - невозрастает, а последовательность $m_n$ - неубывает. Поэтому для \underline{\text{ограниченной}} последовательности существует:\[
	\lowlim\limits_{n\rightarrow\infty} a_n := \lim\limits_{n\rightarrow\infty} m_n - \textbf{нижний частичный предел}
	\] \[
	\uplim\limits_{n\rightarrow\infty} a_n := \lim\limits_{n\rightarrow\infty} M_n - \textbf{верхний частичный предел}.
	\]
	\subsection{Теорема №1}
	\label{subsec:t1}
	\textit{Пусть} $\{a_n\}_{n=1}^{\infty}$ \textit{-- ограниченная последовательность. Тогда} $\uplim\limits_{n \to \infty} a_n, \lowlim\limits_{n \to \infty} a_n$ \textit{-- частичные пределы последовательности} $\{a_n\}_{n=1}^{\infty}$ \textit{и любой другой предел принадлежит отрезку} \[[\lowlim\limits_{n \to \infty} a_n, \uplim\limits_{n \to \infty} a_n]
	\]
	\textit{Доказательство.}
	Покажем, что $M := \uplim\limits_{n \to \infty} a_n$ -- частичный предел. Для этого индуктивно построим последовательность, которая сходится к $\uplim\limits_{n \to \infty} a_n$. Пусть $n_1 = 1$. Пусть индексы $n_1 < n_2 < ... < n_k$ уже построены. Тогда подберём такой номер $n_{k+1} > n_k$, что
	\[
	M_{n_k} - \frac{1}{k+1} < a_{n_{k+1}} \leq M_{n_k}.
	\]
	Как подпоследовательность сходящейся последовательности $M_{n_k} \to M$, поэтому по теореме о сходимости зажатой последовательности (по теореме о двух полицейских и преступнике) получаем, что $\lim\limits_{k \to \infty} a_{n_k} = M$. \\\\
	Аналогично проверяется и то, что $\lowlim\limits_{n \to \infty} a_n$ -- частичный предел. \\\\
	Пусть теперь $a$ -- частичный предел. Это означает, что $a = \lim\limits_{k \to \infty} a_{n_k}$ для некоторой подпоследовательности $\{a_{n_k}\}_{n=1}^{\infty}$. Тогда $m_{n_{k-1}} \leq a_{n_{k}} \leq M_{n_{k-1}}$. По теореме о переходе к пределу в неравенствах получаем, что $\lowlim\limits_{n \to \infty} a_n \leq a \leq \uplim\limits_{n \to \infty} a_n$..
	\subsection{Следствие из Теоремы 1}
	Теорема Больцано - во всякой ограниченной последовательности можно найти сходящуюся подпоследовательность.
	\subsection{Теорема №2}
	$\textit{Ограниченная последовательность сходится тогда, и только тогда}, \textit{когда множество её частичных}\\ \textit{пределов  состоит из одного элемента.}$\\\\
	$\textit{Докозательство}$. То, что у сходящейся последовательности только один предел, уже доказано ранее.\\
	Теперь предположим, что у $\underline{\text{ограниченной}}$ последовательности $\left\{a_n\right\}_{n=1}^{\infty}$ только один частичный предел. По доказаному в \hyperref[subsec:t1]{Теорема №1} в частности это означает, что \[
	\lowlim\limits_{n\rightarrow\infty} a_n = \uplim\limits_{n\rightarrow\infty} a_n  = A
	\]
	Тогда, $m_{n-1} \leqslant a_n \leqslant M_{n-1}$, и по теореме о сходимости зажатой последовательности, получаем что $\lim\limits_{n\rightarrow\infty}a_n = A$\textbf{}

\section{Теорема Больцано. Фундаментальная последовательность и критерий Коши. Расходимость последовательности $a_n = \sum^n_{k=1}\frac{1}{k}$. Вычисление $\sqrt{2}$ с помощью рекуррентной формулы $a_{n+1} = 1 + \frac{1}{1+a_n}, a_1 = 1$, обоснование сходимости.}
\subsection{Теорема Больцано}
(Следствие теоремы о связи верхнего и нижнего частичного предела с множеством частичных пределов. \textit{Теорема 23}) \\ 
Во всекой ограниченной последовательности можно найти сходящуюся подпоследовательность. ($\{a_n\}^{\infty}_{n=1}-\text{ограниченная} \Rightarrow \exists$ сходящаяся подпоследовательность.)
\subsection{Фундаментальная последовательность и критерий Коши}
Говорят, что последовательность $\{a_n\}^\infty_{n=1}$ фундаментальна (или является последовательностью Коши), если для каждого числа $\varepsilon > 0$ найдется такое натуральное число (номер) $N(\varepsilon) \in \mathds{N}$, что $|a_n - a_m| < \varepsilon$ при каждых $n, m > N(\varepsilon)$. То же самое утверждение можно переписать в кванторах $\forall - \exists$ следующим образом: $$\forall \varepsilon > 0 \exists N(\varepsilon) \in \mathds{N}: \forall n, m > N(\varepsilon) \ |a_n - a_m| < \varepsilon$$ \\ 
\textbf{Пример.} \begin{itemize}
    \item[1)]Последоввательность $a_n =\frac{1}{n}$  фундаментальная. Действительно $$\left|\frac{1}{n} - \frac{1}{m}\right| \leqslant max\left\{\frac{1}{n}, \frac{1}{m}\right\},$$ поэтому при $N(\varepsilon) = \left[\frac{1}{\varepsilon}\right] + 1 > \frac{1}{\varepsilon} \text{ выполнено } |a_n - a_m| < \varepsilon$
    \item[2)] Последовательность $a_n = (-1)^n$ не фундаментальная. Действительно, если мы возьмем  $\varepsilon = 1$, то, какой бы ни был номер $N(\varepsilon)$, для произвольного $n > N(\varepsilon)$ выполнено $|a_n - a_{n+1}| = 2 > 1.$
\end{itemize} 
\textbf{Критерий Коши}. $\{a_n\}_{n=1}^\infty$ - сх-ся $\iff$ $\{a_n\}_{n=1}^\infty$ - посл. Коши
\begin{proof}
$\Longrightarrow$ Пусть $\varepsilon > 0$ По определению сходящейся последовательности найдется такой номер $N \in \mathds{N}$, что $|a_n - a| < \frac{\varepsilon}{2}$ при $n > N$, где $ \lim_{n\to\infty} a_n = a$. Тогда при $n, m > N$ выполнено $$|a_n - a_m| = |a_n - a_m + a - a| \leqslant |a_n - a| + |a_m - a| < \varepsilon$$ \\ $\Longleftarrow$(План: 1. Ограничена 2. предел по т. Больцано 3. $a = \lim_{n\to\infty}a_n$) \begin{itemize}
    \item[1.] Заметим, что последовательность $\{a_n\}_{n=1}^\infty$ ограничена. $\varepsilon = 1 \ \exists N: \forall n, m > N: |a_n - a_m| < 1$ (из условия). Отсюда $|a_n| = |a_n + a_{N+1} - a_{N+1}| \leqslant |a_n - a_{N+1}| + |a_{N+1}| < 1 + |a_{N+1}|$, при $n>N$. Значит, $$|a_n| < M = max\{1+|a_{N+1}|, |a_1|, ..., |a_N|\}.$$
    \item[2.]У ограниченной последовательности $\{a_n\}_{n=1}^\infty$ по теореме Больцано есть хотя бы один частичный предел $a$. $\Rightarrow \exists \{a_{n_{k}}\}_{k=1}^\infty: a_{n_{k}} \rightarrow a$
    \item[3.] $\forall \varepsilon > 0\ \exists k_0: k > k_0\ |a_{n_{k}} - a| < \varepsilon.$ Кроме того, в силу фундаментальности найдется номер N, для которого $|a_n - a_m| < \varepsilon$ при $n, m > N$. Пусть $k$ выбрано так, что $k > k_0$ и $n_k > N$, тогда при каждом $n > N$ выполнено, что $$|a_n - a| = |a_n + a_{n_{k}} - a_{n_{k}} - a| < \underset{<\varepsilon}{|a_n - a_{n_k}|} + \underset{<\varepsilon}{|a- a_{n_k}|} < 2\varepsilon$$
\end{itemize}
\end{proof}
\subsection{Расходимость последовательности $a_n = \sum^n_{k=1}\frac{1}{k}$} Проверим отридцание фундаментальности $$\exists \varepsilon > 0 \ \forall N: \exists \underset{n > m}{n,m > N}:|a_n - a_m| > \varepsilon$$ \\ $|a_n - a_m| = \frac{1}{m+1} + \frac{1}{m+2}+...+\frac{1}{n} 
\geqslant \frac{n-m}{n}= 1 - \frac{m}{n}$ Для $\varepsilon = \frac{1}{2}, n = 2m, m > N \Longrightarrow |a_n - a_m| \leqslant \frac{1}{2} \Longrightarrow$ не выполнено условие Коши $\Longrightarrow$ последовательность расходится
\subsection{Вычисление $\sqrt{2}$ с помощью рекуррентной формулы $a_{n+1} = 1 + \frac{1}{1+a_n}, a_1 = 1$, обоснование сходимости} Заметим, что $a_n \geqslant 1$ и $$|a_{n+1} - a_n| = \left|\frac{1}{1+a_n} - \frac{1}{1+a_{n-1}} \right| = \frac{|a_{n-1} - a_n|}{(1+a_n)(1+a_{n+1})} \leqslant \frac{1}{4}|a_{n-1} - a_n| \leqslant \left(\frac{1}{4}\right)^{n-1}(a_2 - a_1) = \left(\frac{1}{4}\right)^{n-1} \frac{1}{2}$$ Отсюда при $m > n$ $$|a_m - a_n| \leqslant |a_m - a_{m-1}| +...+ |a_{n+1} - a_n| \leqslant \frac{1}{2}\left(\left(\frac{1}{4}\right)^{m-2} + ... + \left(\frac{1}{4}\right)^{n-1}\right) = \frac{1}{2}\left(\frac{1}{4}\right)^{n-1}\left(\frac{1 - \left(\frac{1}{4}\right)^{m-n}}{1 - \frac{1}{4}}\right) = \frac{8}{3}\left(\frac{1}{4}\right)^n$$ Т.к. $\left(\frac{1}{4}\right)^n \rightarrow 0: \forall \varepsilon > 0 \ \exists N: n> N \ \left(\frac{1}{4}\right)^n < \varepsilon$. Тем самым, для последовательности ${an}_{n=1}^\infty$ выполнен критерий Коши, а значит существует $A = \lim_{n\to\infty} a_n \Longrightarrow \lim_{n\to\infty} a_n = A = \lim_{n\to\infty} a_{n+1} = 1 + \frac{1}{1 + A} \\ A(A+1) = A + 1 + 1 \iff A^2 = 2 \iff A = \sqrt{2}$ т.к. $a_n \geqslant 0$

\section{Числовые ряды}

\subsection{Числовой ряд}
Пусть дана последовательность $\textstyle\{a_n\}_{n=1}^{\infty}$, тогда числовым рядом с членами $a_n$ называется выражение:
$$a_1 + a_2 + a_3 + ... = \displaystyle\sum_{k = 1}^{\infty}a_k$$

\subsection{Переформулировка критерия Коши для числовых рядов}
Ряд сходится тогда и только тогда, когда:
$$\forall \varepsilon > 0 \ \exists N \in \mathbb{N} : \forall n > m > N \to \left|\displaystyle\sum_{k = m + 1}^n a_k\right| = |S_n - S_m| < \varepsilon$$

\subsection{Необходимое условие сходимости числового ряда}
Если числовой ряд сходится, то $a_k \to 0$ при $k \to \infty $
\begin{proof}
Из критерия коши следует, что: $$\forall \varepsilon > 0 \ \exists N \in \mathbb{N}: \forall n >  N + 1 \to |a_n| = |S_n - S_{n-1}| < \varepsilon$$
\end{proof}

\subsection{Абсолютная и условная сходимость рядов}
% \begin{enumerate}
Говорят, что ряд $\displaystyle\sum_{k = 1}^{\infty}a_k$ сходится абсолютно, если сходится ряд $\displaystyle\sum_{k = 1}^{\infty}|a_k|$ \\
Говорят, что ряд $\displaystyle\sum_{k = 1}^{\infty}a_k$ сходится условно, если он сходится, а ряд $\displaystyle\sum_{k = 1}^{\infty}|a_k|$ расходится 
% \end{enumerate}

\subsection{Cходимость абсолютно сходящегося ряда}
Если ряд $\displaystyle\sum_{k = 1}^{\infty}|a_k|$ сходится, то и $\displaystyle\sum_{k = 1}^{\infty}a_k$ тоже сходится
\begin{proof}
Из сходимости ряда $\displaystyle\sum_{k = 1}^{\infty}|a_k|$ следует выполнение критерия Коши для этого ряда, то есть что:
$$\forall \varepsilon > 0 \ \exists N \in \mathbb{N}: \forall n > m > N \to \displaystyle\sum_{k = m + 1}^{n}|a_k| < \varepsilon$$
но так как $\displaystyle\sum_{k = m + 1}^{n}|a_k| \geq \left|\displaystyle\sum_{k = m + 1}^{n}a_k\right|$, то критерий Коши выполнен и для ряда без модулей.
\end{proof}

\subsection{Признак сравнения}
% \begin{enumerate}
Пусть $0 \leq a_n \leq b_n$ тогда если ряд $\displaystyle\sum_{k = 1}^{\infty}b_k$ сходится, то и $\displaystyle\sum_{k = 1}^{\infty}a_k$ сходится.\\
Если же $\displaystyle\sum_{k = 1}^{\infty}a_k$ расходится, то и $\displaystyle\sum_{k = 1}^{\infty}b_k$ расходится.
% \end{enumerate}

\subsection{Признак Коши}
Пусть $\textstyle\{a_n\}_{n=1}^{\infty}$ - невозрастающая последовательность, $ a_n \geq 0$. Ряд $\displaystyle\sum_{k = 1}^{\infty}a_k$ сходится тогда и только тогда, когда сходится ряд $\displaystyle\sum_{k = 1}^{\infty}2^k a_{2^k}$
\begin{proof}
Заметим, что $a_2 + 2a_3 + ... + 2^{n-1}a_{2^n} \leq a_2 + a_3 + ... + a_{n^k} \leq 2a_2+ 4a_4 + ... + 2^n a_{2^n}$, тогда из ограниченности частичных сумм ряда $
\displaystyle\sum_{k = 1}^{\infty}2^k a_{2^k}$ следует ограниченность частичных сумм $\displaystyle\sum_{k = 1}^{\infty}a_k$ и наоборот
\end{proof}

\subsection{Сходимость ряда $\displaystyle\sum_{k = 1}^{\infty}\frac{1}{k^p}$}
Ряд $\displaystyle\sum_{k = 1}^{\infty}\frac{1}{k^p}$ сходится при $p > 1$ и расходится при $p \leq 1$
\begin{proof}
При $p < 0$ слагаемое $\frac{1}{k^p}$ не стремится к нулю следовательно ряд расходится.
При $p > 0$: по признаку Коши ряд сходится тогда и только тогда, когда сходится и ряд $\displaystyle\sum_{k = 1}^{\infty}\frac{2^k}{2^{kp}} = \displaystyle\sum_{k = 1}^{\infty}(2^{1-p})^k$, а это сумма геометрической прогрессии, которая сходится при $2^{1-p} < 1$, то есть при $p > 1$ и расходится при $p \leq 1$
\end{proof}

\section{Открытые и замкнутые множества на прямой, их свойства, связанные с теоретико-множественными операциями. Внутренние, предельные и граничные точки множеств. Четыре эквивалентных описания замкнутого множества.}

\subsection{Открытые и замкнутые множества на прямой, их свойства, связанные с теоретико-множественными операциями.}

\textbf{Определение 1}.

$\varepsilon$-окрестность точки $a \in\mathbb{R}$ называется множество $B_\varepsilon(a) := \{x\in\mathbb{R}: |x-a| < \varepsilon\} = (a-\varepsilon, a +\varepsilon).$

\textbf{Определение 2}.

Проколотая $\varepsilon$-окрестность точки $a \in\mathbb{R}$ называется множество $B'_\varepsilon(a) := B_\varepsilon \backslash \{a\}$.

\textbf{Определение 3}.

Множество $U\subset\mathbb{R}$ называется \textbf{открытым}, если для любой $a\in U$ найдётся такое $\varepsilon > 0$, что $B_\varepsilon \subset U$.

\textbf{Определение 4}.

Множество $V \subset\mathbb{R}$ называется \textbf{замкнутым}, если его дополнение открыто, т.е. $\mathbb{R} \backslash V - $ открытое множество.

\textbf{Пример}.

\begin{itemize}
    \item[1.] Всякий интервал $(\alpha,\beta)$ - открытое множество, т.к. для каждой точки $a \in (\alpha, \beta)$ множество $B_{\min\{\frac{a-\alpha}{2}, \frac{\beta-a}{2}\}} \subset (\alpha,\beta)$. А также вся числовая прямая, лучи $(-\infty, \alpha), (\beta, +\infty)$, пустое множество будут являться открытыми.
    \item[2.] Отрезок $[\alpha, \beta]$, вся числовая прямая, лучи $(-\infty, \alpha], [\beta, +\infty)$, пустое множество будут замкнутыми. (Если попросят доказать что-то отсюда - скажите, что дополнение будет открытым множеством).
\end{itemize}

\textbf{Свойства}.

\textit{Объединение(1.1) любого набора и пересечение(1.2) конечного набора открытых множеств будет открытым множеством.}

\textit{Пересечение(2.1) любого набора и объединение(2.2) конечного набора замкнутых множеств будет замкнутым множеством.}

\begin{proof}

 

\begin{itemize}
    \item[1.1] Пусть $U = \bigcup\limits_{\alpha\in A}U_\alpha$, причём все $U_\alpha - $ открытые множества. Если $a\in U$, тогда найдётся такой индекс $\alpha$, что $a\in U_\alpha$. По определению найдётся такое $\varepsilon > 0$, что $B_\varepsilon(a) \subset U_\alpha$. Значит, по определению операции объединения, $B_\varepsilon(a)\subset U$. Т.е. $U - $ открытое множество.
    \item[1.2] Пусть $U = \bigcap\limits_{j=1}^NU_j$, причём все $U_j - $ открытые множества. Если $a\in U$, то для каждого \newline$j\in \{1,...,N\}$ найдётся такое число $\varepsilon_j>0$, что $B_{\varepsilon_{j}}(a) \subset U_j$. Пусть $\varepsilon := \min\{\varepsilon_1,...,\varepsilon_N\}>0$. Тогда $B_\varepsilon(a)\subset B_{\varepsilon_j}(a) \subset U_j$ при каждом $j \in \{1,...,N\}$. Значит, $B_\varepsilon(a) \subset U$ и $U - $ открытое множество.
    \item[2.1] Пусть $V = \bigcap\limits_{\alpha\in A}V_\alpha$, причём все $V_\alpha - $ замкнутые множества. По формулам де Моргана\newline $\mathbb{R}\backslash V = \mathbb{R}\backslash \bigcap\limits_{\alpha\in A}V_\alpha = \bigcup\limits_{\alpha\in A}(\mathbb{R}\backslash V_\alpha)$. По определению замкнутого множества, множества $U_\alpha = \mathbb{R} \backslash V_\alpha - $ открыты. По доказанному в 1.1 свойству открытых множеств, множество $\mathbb{R}\backslash V$ также открыто, а значит множество $V - $ замкнуто.
    \item[2.2] Пусть $V = \bigcup\limits_{j=1}^NV_j$, причём все $V_j - $ замкнутые множества. По формулам де Моргана $\mathbb{R}\backslash V = \mathbb{R}\backslash\bigcup\limits_{j=1}^NV_j=\bigcap\limits_{j=1}^N(\mathbb{R}\backslash V_j).$ По определению замкнутого множества, множества $U_j := \mathbb{R}\backslash V_j - $ открыты. По уже доказанному свойству(1.2) открытых множеств, множество $\mathbb{R}\backslash V$ также открыто, а значит множество $V - $ замкнуто. 
\end{itemize}
\end{proof}

\subsection{Внутренние, предельные и граничные точки множеств.}

\textbf{Определение 5.}

Точка $a\in\mathbb{R}$ называется \textbf{внутренней} точкой множества $M$, если она входит в это множество $M$ с некоторой своей окрестностью \underline{полностью} (т.е. $\exists\varepsilon > 0$ : $B_\varepsilon(a)\subset M$).

\textbf{Определение 6.}

Точка $a\in\mathbb{R}$ называется \textbf{предельной} точкой множества $M$, если каждая её проколотая окрестность имеет непустое пересечение с множеством $M$ (т.е. $\forall\varepsilon > 0: B'_\varepsilon(a)\cap M\neq \o$).

\textbf{Определение 7.}

Точка $a\in\mathbb{R}$ называется \textbf{граничной} точкой множества $M$, если каждая её окрестность имеет непустое пересечение как с множеством $M$, так и с его дополнением (т.е. $\forall\varepsilon > 0: B_\varepsilon(a)\cap M\neq \o$ и $B_\varepsilon(a)\cap(\mathbb{R}\backslash M)\neq\o$).

\textbf{Пример.}

Для множества $M = (0, 1]\cup\{3\}$ точки $0, \frac{1}{2}, 1$ будут предельными, а точки $-1, 3$ не будут. Точки $0, 1, 3$ будут граничными, а $-1$ и $\frac{1}{2}$ не будут. Точка $\frac{1}{2}$ будет внутренней, а точки $-1, 0, 1, 3$ не будут.

\textbf{Замечание.}

Точка $a$ предельная для $M$ тогда и только тогда, когда найдётся сходящаяся к $a$ последовательность $a_n \in M\backslash\{a\}$. 

\begin{proof}
Действительно, если $a$ предельная, то для каждого $n$ найдётся точка \newline$a_n\in B'_{1/n}(a)\cap M$. Тогда $a_n\in M\backslash\{a\}$ и $a_n\rightarrow a.$ 

Наоборот (если есть сходящаяся к $a$ последовательность, то $a - $ предельная точка для $M$), если $a_n\in M\backslash\{a\},$ то каждого $\varepsilon>0$ найдётся такой номер $N$, что $|a_n - a|<\varepsilon$ при $n>N$. Таким образом, $a_{N+1}\in B'_\varepsilon(a)\cap M$.
\end{proof}

\newpage

\subsection{Четыре эквивалентных описания замкнутого множества.}

\textbf{Теорема}

\textit{Следующие утверждения равносильны.}

\begin{itemize}
    \item[1)] \textit{$V$ - замкнутое множество;}
    \item[2)] \textit{$V$ содержит все свои граничные точки;}
    \item[3)] \textit{$V$ содержит все свои предельные точки}
    \item[4)] \textit{если $a_n\in V$ и $a_n\rightarrow a$, то $a\in V$.}
\end{itemize}

\begin{proof}
 
\begin{itemize}
    \item[$1) \Rightarrow2)$]\textit{(Если $V$ - замкнутое множество, то оно содержит все свои граничные точки)}: \newline Пусть $a$ граничная точка для $V$, для которой выполнено, что $a\not\in V$, то $a\in\mathbb{R}\backslash V$ - открытое множество. Это значит, что найдётся такое $\varepsilon > 0$, что $B_\epsilon(a)\subset \mathbb{R}\backslash V$ (т.к. $\mathbb{R}\backslash V$ - открытое множество). Т.е. нашлась окрестность $B_\varepsilon(a)$, которая не пересекается с множеством $V$, а значит $a$ не граничная точка.
    \item[$2) \Rightarrow3)$] \textit{(Если $V$ содержит все свои граничные точки, то оно содежит и все свои предельные)}: \newlineПусть $a$ предельная для $V$ точка и предположим, что $a \not\in V$. Значит $a$ и не граничная точка (т.к. $V$ содержит все свои граничные точки). Поэтому найдётся такое $\varepsilon > 0$, что $B_\varepsilon(a)\cap V = \o$. Таким образом, $B'_\varepsilon(a)\cap V=\o$ и $a$ не предельная для $V$.
    \item[$3) \Rightarrow4)$] \textit{(Если $V$ содержит все свои предельные точки, то если $a_n\in V$ и $a_n\rightarrow a$, то $a\in V$)}: \newline Пусть $a_n\in V, a_n\rightarrow a$. Если $a\not\in V$, то $a\neq a_n$ при каждом $n$. По замечанию выше $a - $ предельная точка для множества $V$, что противоречит тому, что $V$ содержит все свои предельные точки.
    \item[$4) \Rightarrow1)$] \textit{(Если $a_n\in V$ и $a_n\rightarrow a$, то $a\in V$. А отсюда $V - $ замкнутое множество)}: \newline Пусть $V$ - не замкнуто. $\Leftrightarrow$ $\mathbb{R}\backslash V$ - не открыто $\Rightarrow$ существует такое $a\in \mathbb{R}\backslash V: B_\varepsilon(a) \cap V \neq \o$ и при этом $B_\varepsilon(a)\not\subset V$. Тогда пусть $\varepsilon_n = \frac{1}{n} \Rightarrow \exists a_n\in B_{\frac{1}{n}}(a) \cap V \Rightarrow a_n\in V$ и $a_n\rightarrow a \Rightarrow a\in V$ (по условию). Получили противоречие, а значит $V$ - замкнуто.
\end{itemize}
\end{proof}

\section{Компакты на $\mathbb{R}$: определение, 3 базовых свойства. Теорема Бореля-Гейне-Лебега о компактности отрезка. Два эквивалентных описания компактных множеств на $\mathbb{R}$.}

\subsection{Определение.}

\begin{definition}
Говорят, что набор множеств $\{U_\alpha\}_{\alpha\in A}$ образует {\bf покрытие}
множества $M\subset \mathbb{R}$, если $M\subset \bigcup\limits_{\alpha\in A}U_\alpha$
(также гооврят, что система $\{U_\alpha\}_{\alpha\in A}$ является покрытием множества $M$).
\end{definition}

\begin{definition}
Множество $K\subset \mathbb{R}$ называется {\bf компактом} (или компактным множеством),
если для каждого покрытия $\{U_\alpha\}_{\alpha\in A}$ множества $K$
открытыми множествами $U_\alpha$ существует конечный поднабор
$\{U_{\alpha_1}, \ldots, U_{\alpha_N}\}$ этих множеств все еще покрывающий $K$
(т.е. $K\subset \bigcup_{j=1}^N U_{\alpha_{j}}$).

Кратко иногда это свойство формулируют так:
Множество $K$ --- компакт, если из каждого покрытия этого множества открытыми множествами
можно выбрать конечное подпокрытие.
\end{definition}

\subsection{Теорема Бореля-Гейне-Лебега о компактности отрезка.}

\begin{theorem}[Борель--Гейне--Лебег]
Каждый отрезок является компактным множеством.
\end{theorem}

\begin{proof}
Предположим, что есть такой отрезок $[a, b]$ и такое его покрытие $\{U_\alpha\}_{\alpha\in A}$
окрытыми множествами, что никакой конечный поднабор этих множеств не покрывает $[a, b]$.
Рассмотрим подотрезки $[a, \frac{a+b}{2}]$ и $[\frac{a+b}{2}, b]$.
Для какой-то из этих половинок никакой конечный поднабор множеств
$\{U_\alpha\}_{\alpha\in A}$ не покрывает эту половинку (если бы для каждой из половинок
был бы покрывающий ее конечный поднабор,
то и весь отрезок бы покрывался объединением этих конечных поднаборов).
Обозначим эту половинку $[a_1, b_1]$. Снова поделим отрезок пополам
и рассмотрим подотрезки $[a_1, \frac{a_1+b_1}{2}]$ и $[\frac{a_1+b_1}{2}, b_1]$.
Для какой-то из этих половинок никакой конечный поднабор множеств
$\{U_\alpha\}_{\alpha\in A}$ не покрывает эту половинку.
Обозначим эту половинку $[a_2, b_2]$.
Продолжая описанную процедуру индуктивно, строим последовательность вложенных отрезков
$[a_{n+1}, b_{n+1}]\subset [a_n, b_n]$ с тем свойством, что
никакой конечный поднабор множеств
$\{U_\alpha\}_{\alpha\in A}$ не покрывает отрезок $[a_n, b_n]$.

Пусть $c\in \bigcap_{n=1}^\infty[a_n, b_n]$. Т.к. $c\in [a, b]$, то для некоторого
индекса $\alpha$ точка $c\in U_\alpha$. Т.к. $U_\alpha$ --- открытое множество,
то найдется такое число $\varepsilon>0$, что $(c-\varepsilon, c+\varepsilon)\subset U_\alpha$.
Т.к. $b_n-a_n = \frac{b-a}{2^n}\to 0$, то $a_n\to c$ и $b_n\to c\ (|c - a_n|\leq |b_n - a_n|; |c - b_n|\leq |b_n - a_n|)$.
Тогда для некоторого номера $n_0$ выполнено $a_{n_0}\in (c-\varepsilon, c]$
и $b_{n_0}\in [c, c+\varepsilon)$. Т.е.
$[a_{n_0}, b_{n_0}]\subset (c-\varepsilon, c+\varepsilon)\subset U_\alpha$,
что противоречит построению отрезков $[a_n, b_n]$.
\end{proof} 
\textbf{Пример.} $(0, 1)$ -- не компакт. \\ $\left((0, 1-\frac{1}{n})\right) = U_n \\ (0,1) \subset \bigcup_{k=1}^\infty U_k \\ U_{k_1} \cup U_{k_2} \cup ... \cup U_{k_n} = \left(0, 1 - \frac{1}{max(k_1, k2, ..., k_n)}\right)$ (предъявили покрытие, для которого не существует конечный набор множеств, все еще покрывающий $(0, 1)$)
\subsection{3 базовых свойства компактных множеств.}

\begin{lemma} Пусть $K$ --- компакт. Тогда

$1)\ K$ --- ограниченное множество;

$2)\ K$ --- замкнутое множество;

$3)$ замкнутое подмножество $K$ также компактно.
\end{lemma}

\begin{proof}
$1)$ Заметим, что $K\subset \bigcup_{n=1}^\infty(-n ,n)$. Т.к. $K$ --- компакт,
то у данного покрытия найдется конечное подпокрытие, т.е.
$K\subset \bigcup_{j=1}^m(-n_j ,n_j)$. Пусть $C:=\max\{n_1, \ldots, n_m\}$.
Тогда $K\subset (-C, C)$.

$2)$ Пусть $a\in \mathbb{R}\setminus K$.
Тогда $K\subset \bigcup_{n=1}^\infty U_n$, где
$U_n:=(-\infty, a-\frac{1}{n})\cup (a+\frac{1}{n}, +\infty)$.
Выбрав конечное подпокрытие, получаем, что
$K\subset \bigcup_{j=1}^m U_{n_j}$. Пусть $C:=\max\{n_1, \ldots, n_m\}$.
Тогда $K\subset (-\infty, a-\frac{1}{C})\cup(a+\frac{1}{C}, +\infty)$
и $B_{1/C}(a)\subset \mathbb{R}\setminus K$.

$3)$ Пусть $V\subset K$, $V$ --- замкнутое множество.
Пусть $\{U_\alpha\}_{\alpha\in A}$ --- покрытие множества $V$.
Тогда набор, состоящий из множеств $\{U_\alpha\}_{\alpha\in A}$ и $\mathbb{R}\setminus V$
будет покрытием множества $K$ открытыми множествами.
В нем можно найти конечный поднабор $U_{\alpha_1}, \ldots, U_{\alpha_m}$
и, возможно, $\mathbb{R}\setminus V$, покрывающий множество $K$.
Тогда множество $V$ заведомо покрывается набором $U_{\alpha_1}, \ldots, U_{\alpha_m}$.
\end{proof}

\subsection{Два эквивалентных описания компактных множеств на $\mathbb{R}$.}

\begin{corollary}
Множество $K\subset \mathbb{R}$ компактно тогда и только тогда, когда оно ограничено и замкнуто.
\end{corollary}

\begin{proof}
Компактные множества обязаны быть замкнутыми и ограниченными.
Наоборот, если $K$ ограниченное множество, то $K\subset [-C, C]$ для некоторого числа $C>0$.
Т.к. отрезок --- компактное множество, а $K$ --- замкнутое множество,
то $K$ также будет компактным множеством по предыдущей лемме.
\end{proof}

\begin{corollary}
Множество $K\subset \mathbb{R}$ компактно тогда и только тогда, когда
из каждой последовательности элементов этого множества
можно выбрать подпоследовательность, сходящуюся к элементу этого множества.
\end{corollary}

\begin{proof}
Если множество $K$ --- компактно, то оно замкнуто и ограничено.
Пусть $\{a_n\}_{n=1}^\infty\subset K$. По теореме Больцано в данной последовательности
найдется сходящаяся подпоследовательность $a_{n_k}\to a$. В силу замкнутости множества $K$
получаем, что $a\in K$ (см. теорему 10.3).

Наоборот, пусть из каждой последовательности элементов множества $K$
можно выбрать подпоследовательность, сходящуюся к элементу этого множества.
Если бы множество $K$ не являлось ограниченным, то для каждого $n\in \mathbb{N}$
была бы точка $a_n\in K$, $|a_n|>n$. Из такой последовательности невозможно выбрать
сходящуюся подпоследовательность. Пусть теперь $a_n\in K$, $a_n\to a$.
По условию, из этой последовательности можно выбрать подпоследовательность $a_{n_k}$,
сходящуюся к точке множества $K$, т.е. $a_{n_k}\to b\in K$. В силу единственности предела
и совпадения предела подпоследовательности с пределом всей последовательности получаем,
что $a=b\in K$.
\end{proof}

\section{Определения предела функции (по множеству) по Коши и по Гейне, их эквивалентность. Свойства предела функции (единственность, линейность, предел произведения и отношения, предел и неравенства, ограниченность, отделимость, предел композиции). Замечательные пределы.}
\subsection{Предел функции по Коши}
Пусть функция $f$ определена на некотором множестве $D \subset R$ и $a $ - предельная для $D$ точка, тогда
\begin{center}
$\lim_{x \to a} f(x) = A \Longleftrightarrow \\ \forall \epsilon > 0 \ \exists \delta > 0: \forall x \in D \cap \beta'_{\delta}(a) \to |f(x) - A | < \epsilon$
\end{center}
\subsection{Предел функции по Гейне}
Пусть функция $f$ определена на некотором множестве $D \subset R$ и $a $ - предельная для $D$ точка, тогда
\begin{center}
$\lim_{x \to a} f(x) = A \Longleftrightarrow \\ \forall \{x_n\} \in D \backslash \{a\} :\lim_{n \to \infty}x_n = a \to \lim_{n\to \infty} f(x_n) = A$
\end{center}
\subsection{Эквивалентность двух определений}
От Коши к Гейне:\newline Пусть $\lim_{x\to a}f(x) = A$ в смысле коши, тогда рассмотрим последовательность точек $x_n \in D \backslash \{a\}: x_n \to a$, по определению предела по коши 
\begin{center}
$\forall \epsilon > 0 \ \exists \delta > 0: \forall x \in D \cap \beta'_{\delta}(a) \to |f(x) - A | < \epsilon$
\end{center}
последовательность $x_n \to a$, то есть 
\begin{center}
$\exists N: \forall n > N \to x_n \in \beta_{\delta}(a)$    
\end{center}
при $n > N$ $x_n \in D \backslash \{a\} \cap \beta_{\delta}(a)$, то есть при $n > N$ выполняется $
|f(x_n) - A| < \epsilon, $ что и означает, что $A$ - предел функции по гейне\newline
От Гейне к Коши:\newline Пусть число $A$ не является пределом функции $f$ в точке $a$ в смысле коши, тогда это означает, что
\begin{center}
$\exists \epsilon > 0: \forall \delta > 0 \ \exists x_{\delta} \in D \cap \beta'_{\delta}(a): |f(x_{\delta}) - A| \geq \epsilon$
\end{center}
по определению  по гейне: для последовательности точек $ x_{1/n} \in D \backslash \{a\} $ (то есть берем $\delta = 1/n$) выполняется, что $\{x_{1/n}\} \to a$ но заметим, что $|f(x_{1/n}) - A| \geq \epsilon $, тогда $A$ не является пределом $f$ в смысле гейне
\subsection{Свойства}
Пусть функции $f, g, h$ определены на некотором множестве $D \subset R$ и пусть $a$ - предельная для $D$ точка, тогда выполнены следующие свойства:\newline
$\bullet$ Единственность:
\begin{center}
$\lim_{x \to a} f(x) = A, \lim_{x \to a} f(x) = B \Rightarrow A  = B$
\end{center}
$\bullet$ Линейность:
\begin{center}
$\lim_{x \to a} f(x) = A, \lim_{x \to a} g(x) = B \Rightarrow \\\lim_{x \to a} (\alpha f(x) + \beta g(x) ) = \alpha A + \beta B \ \\ \forall \alpha, \beta \in R$
\end{center}
$\bullet$ Предел произведения:
\begin{center}
$\lim_{x \to a} f(x) = A, \lim_{x \to a} g(x) = B \Rightarrow \lim_{x \to a} (f(x) \cdot g(x)) = A \cdot B$
\end{center}
$\bullet$ Предел частного:
\begin{center}
$\lim_{x \to a} f(x) = A, \lim_{x \to a} g(x) = B \not = 0, \forall x \in D \to g(x) \not = 0 \Rightarrow \\
\lim_{x \to a} \frac{f(x)}{g(x)} = \frac{A}{B}$
\end{center}
$\bullet$ Предел и неравенства: (входит ли сюда лемма о милиционерах или нет? Жду ответ Музы)
\begin{center}
$\exists \delta  > 0: \  \forall x \in D \cap \beta'_{\delta}(a) \to  f(x) \leq g(x), \\ \lim_{x \to a} f(x) = A, \lim_{x \to a} g(x) = B \Rightarrow \\A \leq B$
\end{center}
$\bullet$ Ограниченность:
\begin{center}
$\lim_{x \to a} f(x) = A \Rightarrow \exists \delta > 0, C > 0: \forall x \in D \cap \beta'_{\delta}(a)  \to |f(x)| \leq C $
\end{center}
$\bullet$ Отделимость:
\begin{center}
$\lim_{x \to a} f(x) = A \Rightarrow \exists \delta > 0: \forall x \in D \cap \beta'_{\delta}(a)  \to |f(x)| > \frac{|A|}{2}$
\end{center}
Все свойства кроме отделимости и ограниченности следуют из аналогичных свойств для предела последовательности и определения через Гейне (так написано в учебнике).\newline
$\bullet$ Доказательство ограниченности: найдется такое  $\delta > 0: |f(x) - A| < 1$ при $x \in D \cap \beta'_{\delta}(a)$ , таким образом при $x \in D \cap \beta'_{\delta}(a)$ выполнено $|f(x) | < 1 + |A|$\newline
$\bullet$ Доказательство отделимости: найдется такое  $\delta > 0: |f(x) - A| < \frac{|A|}{2}$ при $x \in D \cap \beta'_{\delta}(a)$ , таким образом при $x \in D \cap \beta'_{\delta}(a)$ будет выполнено 
$$
|A| - |f(x)| \leq |f(x) - A| < \frac{|A|}{2} \Rightarrow \\ |f(x)| > \frac{|A|}{2}
$$\newline
$\bullet$ Предел композиции:\newline
Пусть $f: D \xrightarrow{} E, g: E \xrightarrow{} R, a - $ предельная точка множества $D$, $b$ - предельная точка множества E, $\lim_{x \to a} f(x) = b, \lim_{y \to b} g(y) = c$ и есть такая проколотая окрестность $\beta'_{\delta}(a)$ точки $a$, что $f(x) \not = b $ для каждой точки $x \in D \cap \beta'_{\delta}(a)$. Тогда $\lim_{x \to a} g(f(x)) = c$\newline
Доказательство:\newline Пусть $x_n \xrightarrow{} a, x_n \in D$. $ x_n \not = a.$ Т.к. $f(x) \not = b$ для каждой точки $x \in D \cap \beta'_{\delta}(a)$, то найдется такой номер $N$, что $f(x_n) \not = b$ при $n > N_0$. Поэтому последовательность $f(x_{N+1}), f(x_{N+2}), ...$ состоит из элементов множества $E$, ни один из этих элементов не совпадает с $b$ и эта последовательность сходится к $b$. Поэтому последовательность $g(f(x_{N+1})), g(f(x_{N+2})), ... $ сходится к $c$. Значит и вся последовательность \{$g(f(x_n))$\} сходится к $c$  \newline
\subsection{Замечательные пределы}
$\bullet$ Первый замечательный предел: $\lim_{x \to 0}\frac{\sin x}{x} = 1$ \newline
Доказательство: для $x \in (0, \pi/2)$ рассмотрим два треугольника и площадь сектора, сравним их и получим (сначала треугольник внутри круга, потом сектор, потом треугольники со стороной по касательной к кругу)
\begin{center}
$\frac12 \cdot 1 \cdot \sin x \leq \frac{x}2\leq \frac12 \cdot 1 \cdot \tg x$
\end{center}
откуда, в силу четности при $x \in (-\pi/2, \pi/2), x \not = 0$ выполнено
\begin{center}
$\cos x \leq \frac{\sin x}{x} \leq 1$
\end{center}
Утверждение теперь следует из теоремы о зажатой функции, т.к. $\lim_{x \to y} \cos x = \cos y$\newline
Действительно, $|\cos x - \cos y| = 2|\sin{(\frac{x+y}{2})}\sin{(\frac{x-y}{2})}| \leq 2|\sin{(\frac{x-y}{2})}|\leq |x - y|$\newline\newline
$\bullet$ Второй замечательный предел: $\lim _{x \to +\infty}\left(1 + \frac1x\right)^x = e$\newline
Доказательство: рассмотрим функции $f(x) := (1 + \frac{1}{[x] + 1})^{[x]}, g(x) := (1 + \frac{1}{[x]})^{[x] + 1}, $ тогда $ \newline
f(x) \leq \left(1 + \frac1x\right)^x \leq g(x), $ кроме того, т.к.  $\lim_{n \to +\infty} (1 + \frac{1}{n + 1})^{n} = \lim_{n \to +\infty} (1 + \frac{1}{n})^{n + 1} = e$, то и $\lim_{x \to +\infty} f(x) = \lim_{x \to +\infty} g(x) =e$. Утверждение теперь следует из теореме о пределе зажатой функции.

\section{Критерий Коши существования предела функции. Односторонние пределы и теорема Вейерштрасса о существовани односторонних пределов монотонной ограниченной функции.
}

\smallskip
\subsection{Критерий Коши существования предела функции.}
Теорема 65 (Критерий Коши). Пусть f: D$\to$ R и a предельная точка D. Предел $\lim_{x \to a} f(x)$ существует тогда и только тогда, когда для каждого $\epsilon$ > 0 найдется такое $\delta$ > 0, что для каждых $x, y \in B^{'}_{\delta} (a) \cap D$  выполнено $|f(x) - f(y)| < \epsilon$. \\
Доказательство. Если $\lim_{x \to a} f(x) = A$, то для каждого $\epsilon > 0$ найдется такое $\delta$ > 0, что для произвольной точки $x \in B^{'}_{\delta} (a) \cap D$  выполнено $|f(x) - A| < \epsilon / 2$. Тогда для произвольных точек $x, y \in B^{'}_{\delta} (a) \cap D$ выполнено $|f(x) - f(y)| \leq |f(x) - A| + |A - f(y)| < \epsilon$. \\

Предположим, что выполнено условие Коши. Тогда для произвольной последовательности точек $x_n \in D \backslash$ \{a\}, $x_n \to a$, последовательность $\{f(x_n)\}$ является фундаментальной, а значит сходится. Пусть $\lim_{x \to \infty} f(x_n) = A$. Если есть другая последовательность точек $y_n \in D \backslash \{a\}, y_n \to a$, то рассмотрим новую последовательность $z_{2k1} = x_k, z_{2k} = y_k$, т.е. эта последовательность вида $x_1, y_1, x_2, y_2, \dots \subset D \backslash \{a\}$. Эта последовательность также сходится к a, поэтому последовательность образов $f(x_1), f(y_1), f(x_2), f(y_2), \dots$ снова оказывается фундаментальной, а потому сходится. В силу того, что предел подпоследовательности сходящейся последовательности совпадает с пределом всей последовательности, получаем, что $\lim_{x \to \infty} f(y_n) = A$. Таким образом, доказано существование предела по Гейне.
\smallskip 
\subsection{Односторонние пределы и теорема Вейерштрасса о существовани односторонних пределов монотонной ограниченной функции.} 
Пусть $D^{+}_{a} := D \cap (a, +\infty)$ и $D^{-}_{a} := D \cap (-\infty, a)$\\
Определение 66. Пусть точка a предельная для множества $D^{+}_{a}$ и существует предел функции f по множеству $D^{+}_{a}$ в точке a. Этот предел называют пределом справа функции f в точке a и обозначают $\lim_{x \to a+0} f(x)$. Аналогично определяется предел слева, который обозначают $\lim_{x \to a-0} f(x)$. \\

Теорема 67 (Вейерштрасс). Пусть f не убывает и ограничена на множестве D, a - предельная точка множества $D^{-}_{a}$. Тогда существует предел слева
$$\lim_{x \to a-0} f(x) = sup\{f(x):x \in D^{-}_{a}\}$$
Пусть f не убывает и ограничена на множестве D, a предельная точка множества $D^{+}_{a}$. $$\lim_{x \to a+0} f(x) = inf\{f(x):x \in D^{+}_{a}\}$$
Аналогичные утверждения с заменой inf на sup справедливы и для невозрастающей
функции.\\
Доказательство. Пусть M = $sup\{f(x):x \in D^{-}_{a}$. Тогда для каждого $\epsilon$ > 0 найдется такая точка $x_0 \in D^{-}_{a}$, что $M-\epsilon < f(x_0)$. Т.к. f не убывает на $D^{-}_{a}$, то для каждого $x \in (x_0, a) \cap D^{'}_{a} $ выполнено $M - \epsilon < f(x_0) \leq f(x) \leq M < M + \epsilon$. Тогда, взяв $\delta := a - x_0$ получаем, что для каждого $x \in B^{'}_{\delta}(a) \cap D^{-}_{a}$ выполнено $|f(x) - M| < \epsilon$.


\end{document}
