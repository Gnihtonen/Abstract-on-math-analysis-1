\documentclass[12pt]{article}
\usepackage[utf8]{inputenc}
\usepackage[russian]{babel}
\usepackage[left=1.5cm,right=1.5cm,top=1.5cm,bottom=2cm]{geometry}
\usepackage{graphicx}
\usepackage{color}
\usepackage{titlesec}
\usepackage{amssymb}
\usepackage{mathtools}
\usepackage{minted}
\usepackage{hyperref}
\usepackage{amsmath}
\usepackage{amsfonts}
\usepackage{fancyhdr}
\usepackage{enumitem}
\usepackage{mathbbol}
\usepackage{titlesec}
\usepackage{changepage}
\usepackage{listings}
\usepackage[titles]{tocloft}
\usepackage{tcolorbox}
\usepackage{tikz}
\usepackage{tikz-cd}
\usepackage{ulem}
\usepackage{nicefrac}
\usepackage{indentfirst}
\usepackage{dsfont}
\usepackage{amsthm}

\newtheorem{theorem}{Теорема}%[section]
\newtheorem{lemma}[theorem]{Лемма}
\newtheorem{corollary}[theorem]{Следствие}
\newtheorem{remark}[theorem]{Замечание}
\newtheorem{example}[theorem]{Пример}
\newtheorem{proposition}[theorem]{Предложение}
\theoremstyle{definition}
\newtheorem{definition}[theorem]{Определение}

\usepackage{hyperref}
\hypersetup{
    colorlinks=true,
    linkcolor=cyan,
    filecolor=magenta,      
    urlcolor=blue,
    pdftitle={Overleaf Example},
    pdfpagemode=FullScreen,
    }

\title{Коллоквиум по курсу "Математический анализ", I курс, осенний семестр 2022}
\author{Группа БПМИ2211}

\begin{document}
\maketitle

\section{Вычисление $\sqrt{2}$ с помощью рекурентной формулы $a_{n+1}=\frac{1}{2}\bigl(a_n + \frac{2}{a_n}\bigr), a_1 = 2$, обоснование сходимости и оценка скорости сходимости. Число $e$ (определение и обоснование корректности).}
\subsection{Вычисление $\sqrt{2}$ с помощью рекурентной формулы $a_{n+1}=\frac{1}{2}\bigl(a_n + \frac{2}{a_n}\bigr)$}
{\rm 
$(\sqrt{a} - \sqrt{b})^2 \geq 0, \  a - 2\sqrt{ab} + b \geq 0, \ a + b \geq 2\sqrt{ab}$ \\ 
Заметим, что 
$$
a_{n+1}=\frac{1}{2}\Bigl(a_n + \frac{2}{a_n}\Bigr)\ge \frac{1}{2}\cdot 2\sqrt{a_n\cdot\frac{2}{a_n}}=\sqrt2.
$$
Поэтому $a_n\ge\sqrt 2$.
Кроме того
$
a_{n+1}=\frac{1}{2}\bigl(a_n + \frac{2}{a_n}\bigr) \le \frac{1}{2}\bigl(a_n + \frac{a_n^2}{a_n}\bigr)=a_n.
$ \\ 
$\{a_n\}_{n=1}^\infty$ ограничена снизу и не возрастает, тогда по т. Вейерштрасса у последовательности $\{a_n\}_{n=1}^\infty$
существует предел $a$. Т.к. $a_n\ge \sqrt2>0$, то и $a> 0$.
Тогда, по арифметике предела получаем
$a =\frac{1}{2}\bigl(a + \frac{2}{a}\bigr)$, откуда $a=\sqrt2$.

Исследуем теперь скорость сходимости:
$$
|a_{n+1} - \sqrt2| = \frac{|a_n^2 - 2 a_n\sqrt2 +2|}{2a_n}
=
\frac{(a_n - \sqrt2)^2}{2a_n}
\le
\frac{(a_n - \sqrt2)^2}{2\sqrt2}\le (a_n - \sqrt2)^2.
$$
Индуктивно получаем
$$
|a_{n+1} - \sqrt2| \le (a_n - \sqrt2)^2 \le (a_{n-1} - \sqrt2)^4
\le (a_{n-2} - \sqrt2)^8\le (a_1 - \sqrt2)^{2^{n}} = (2 - \sqrt2)^{2^{n}}.
$$
Заметим, что $q:=2-\sqrt2 <1$, поэтому полученная скорость сходимость $q^{2^n}$
быстрее экспоненциальной $q^n$
(в смысле количества применений рекуррентной формулы для достижения заданной точности).
}
\subsection{Число $e$}
У последовательности $a_n = \left(1+\frac{1}{n}\right)^n$ есть предел, который называют \textbf{числом} $e$. \\ 
Пусть $a_n = \Bigl(1+\frac{1}{n}\Bigr)^{n}$.
По биному Ньютона
$$
a_n = \sum_{k=0}^{n}C_n^k \frac{1}{n^k}
= 2+\sum_{k=2}^{n}\frac{1}{k!}\frac{n\cdot (n-1)\cdot\ldots\cdot (n-k+1)}{n^k}
=2+\sum_{k=2}^{n}\frac{1}{k!}\Bigl(1-\frac{1}{n}\Bigr)\cdot\ldots\cdot\Bigl(1-\frac{k-1}{n}\Bigr).
$$
Отсюда, во-первых, получаем, что
$$
a_n\le2+\sum_{k=2}^{n}\frac{1}{k!}\le 2+\sum_{k=2}^{n}\frac{1}{2^{k-1}}< 3,
$$
где было использовано неравенство $k! = 1\cdot2\cdot3\cdot\ldots\cdot k \ge 2^{k-1}$
при $k\ge 2$.
Во-вторых,
\begin{multline*}
a_n=2+\sum_{k=2}^{n}\frac{1}{k!}\Bigl(1-\frac{1}{n}\Bigr)\cdot\ldots\cdot\Bigl(1-\frac{k-1}{n}\Bigr)
\le
2+\sum_{k=2}^{n}\frac{1}{k!}\Bigl(1-\frac{1}{n+1}\Bigr)\cdot\ldots\cdot\Bigl(1-\frac{k-1}{n+1}\Bigr)
\\
\le
2+\sum_{k=2}^{n+1}\frac{1}{k!}\Bigl(1-\frac{1}{n+1}\Bigr)\cdot\ldots\cdot\Bigl(1-\frac{k-1}{n+1}\Bigr)=a_{n+1}.
\end{multline*}
Таким образом, последовательность $a_n$ --- неубывает и ограничена сверху, а значит имеет предел,
который называют {\bf числом $e$}.



\end{document}
