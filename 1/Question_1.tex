\documentclass[12pt]{article}
\usepackage[utf8]{inputenc}
\usepackage[T2A]{fontenc}
\usepackage{amsfonts}

\begin{document}

\section*{Билет 1}
\subsubsection*{Рациональные числа}
\textbf{Рациональные числа} – это числа вида $\frac{p}{q}$, где p – целое число, q – натуральное число, причём два числа $\frac{p_1}{q_1}$  и $\frac{p_2}{q_2}$  считаются равными, если $p_1q_2=p_2q_1$. Все свойства натуральных, целых, рациональных чисел и операций над ними будем считать известными.

\subsection*{Десятичные дроби и вещественные числа}

Каждое рациональное число можно представить в виде конечной или бесконечной периодической десятичной дроби, например: $\frac{1}{10}=0.1,   \frac{1}{7}=0.(142857)$. Пусть $0.(9)=x$, тогда $10x=9+x$, значит, $0,(9)=1$, поэтому десятичные записи с периодом 9 рассматривать не будем.

Множество \textbf{вещественных (действительных)} чисел отождествляется с множеством всех десятичных дробей вида $ \pm a_0.a_1 a_2\ldots$,  где $a_0 \in \mathbb {N} \cup \{0\}$, $a_j \in \{0,…,9\}$, и записи, в которых с какого-то момента стоят одни девятки, запрещены. Число $\pm 0.00\ldots$ совпадает с числом 0 и называется нулём. Ненулевое число называется положительным, если в его записи стоит знак + (который обычно опускается). Ненулевое число называется отрицательным, если в его записи стоит знак $-$. B вещественные числа естественным образом вложены рациональные.

На множестве вещественных чисел также определены операции сложения и умножения, для которых справедливы все их естественные свойства (множество вещественных чисел является полем).

На вещественных числах задано \textbf{отношение порядка} следующим образом: на положительных вещественных числах задан лексикографический порядок, т. е. $a_0.a_1a_2\ldots\leq b_0.b_1 b_2\ldots$ тогда и только тогда, когда $a_0.a_1 a_2 \ldots = b_0.b_1 b_2 \ldots $ или найдётся разряд k, для которого $a_0=b_0,\ldots,a_{k-1}=b_{k-1}$ и $a_k<b_k$, который естественным образом переносится на отрицательные.

Для вещественных чисел определён модуль числа $|a|$, равный $-a$ при $a < 0$ и $a$ при $a \geq 0$. Напомним, что для модуля выполнено \textbf{неравенство треугольника} $|a+b| \leq |a|+|b|$. Из неравенства треугольника следует, что $||a|-|b|| \leq |a+b|$.

\subsection*{Принцип полноты}
Будем говорить, что множество чисел A лежит \textbf{левее} множества B, если для каждого $a \in A$ и каждого $b \in B$ выполняется неравенство $a \leq b$. Например, если $A=\{a \in \mathbb{Q}:a<4\}$, $B=\{b \in \mathbb{Q}:b>4\}$, то A левее B.

Если множество A левее множества B, то говорят, что число $c$ \textbf{разделяет} множества A и B, если $a \leq c$ для каждого $a \in A$ и $c \leq b$ для каждого $b \in B$. Например, число 4 разделяет множества A и B, заданные выше.

Будем говорить, что на множестве чисел выполнен \textbf{принцип полноты}, если для произвольных непустых подмножеств A левее B нашего множества найдётся разделяющий их элемент.


\textbf{Теорема.} \textit{На множестве вещественных чисел выполняется принцип полноты.}

\begin{proof} Пусть A и B – непустые множества чисел, причём A левее B. Если A состоит только из неположительных чисел, а B – только из неотрицательных, то нуль разделяет множества A и B.

Предположим, что в A есть положительный элемент, тогда B состоит только из положительных чисел (случай, когда в B есть отрицательное число, рассматривается аналогично). Построим число $c=c_0.c_1 c_2\ldots$, разделяющее A и B.

Рассмотрим множество всех целых неотрицательных чисел, с которых начинаются элементы множества B (это множество состоит из целых неотрицательных чисел в силу того, что в B есть только положительные числа). Пусть $b_0$ – наименьшее из таких чисел и положим $c_0=b_0$. Теперь рассмотрим все числа в множестве B, начинающиеся с $c_0$, и найдём у них наименьшую первую цифру после запятой. Пусть эта цифра $b_1$, тогда полагаем $c_1=b_1$. Теперь рассмотрим все числа в множестве B, начинающиеся с $c_0.c_1$, и найдём у них наименьшую вторую цифру после запятой. Пусть эта цифра $b_2$, тогда полагаем $c_2=b_2$. Аналогично ищутся остальные цифры числа $c$.

Таким образом построена бесконечная десятичная дробь $c_0.c_1 c_2\ldots$. Заметим, что если бы у построенной десятичной записи с какого-то момента шли бы только девятки, то и в B было бы число, в записи которого с какого-то момента участвуют только девятки, но такие записи мы запретили.

Покажем, что построенное число разделяет множества A и B.

Во-первых, по построению $c \leq b$ для каждого $b \in B$. Действительно, либо $b=c$ (тогда всё ОК), либо $b \neq c$. Во втором случае пусть $b_0=c_0,\ldots, b_{k-1}=c_{k-1}$ и $b_k \neq c_k$. Тогда, по построению числа $c$, $c_k<b_k$ и $c<b$.

Покажем, что $a \leq c$ для каждого $a \in A$. Предположим, что $a>c$, т. е. $a \geq c$ и $a\neq c$. Тогда найдётся позиция k, для которой $a_0=c_0,\ldots, a_{k-1}=c_{k-1}$ и $a_k>c_k$. Но по построению числа $c$ есть такой $b\in B$, что $b_0=c_0,\ldots b_k=c_k$, а значит $a>b$, что противоречит условию A левее B.
\end{proof}


\end{document}
